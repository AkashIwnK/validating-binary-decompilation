\begin{figure}
% \centering
% \begin{tabular}{l@{\hskip 0.25in}|@{}l}
\begin{lstlisting}[
language=C++, 
basicstyle=\scriptsize,
keywordstyle=\color{blue},
%numberstyle=\tiny\color{codegray},
%numbers=left,
%xleftmargin=\parindent
]
  int s = 0; int n = N;
  while (n > 0) { s = s + n; n = n - 1; }
  return s;
\end{lstlisting}
%\vspace{-5pt}
\begin{center}
{\small (a) C source code}
\end{center}
% & 
\begin{lstlisting}[
language=Bash,
keywordstyle=\color{blue},
morekeywords={movl,cmpl,jle,addl,decl,jmp,ret},
commentstyle=\color{gray},
basicstyle=\scriptsize,
%numberstyle=\tiny\color{codegray},
breaklines=true,
breakatwhitespace=false,
%numbers=left,
%xleftmargin=\parindent,
stepnumber=1
]
  movl %edi, -20 (%rbp)   # %edi holds N
  movl $0, -4 (%rbp)      # s = 0
  movl -20(%rbp), %eax
  movl %eax, -8(%rbp)     # n = N
  L3: # loop header  
    cmpl $0, -8(%rbp)     # check n <= 0
    jle L2  # if n <= 0, then jump to end, else continue
    movl -8(%rbp), %eax   # n > 0 at this point
    addl %eax, -4(%rbp)   # s = s + n
    decl -8(%rbp)         # n = n - 1
    jmp L3                # jump back to loop header
  L2:
    movl -4(%rbp), %eax   # n <= 0 at this point
    ret
\end{lstlisting}
%\vspace{-5pt}
\begin{center}
{\small (b) \ISA assembly code}
\end{center}
% \multicolumn{1}{c|}{(a)} & \multicolumn{1}{c}{(b)} \\ 
% \multicolumn{1}{c|}{\footnotesize C version} & \multicolumn{1}{c}{\footnotesize \ISA assembly version} \\ \hline
% \end{tabular}
%\vspace{-5pt}
\caption{\s{sum-to-n} program}
%\vspace{-20pt}
\label{fig:sum2n}
\end{figure}

