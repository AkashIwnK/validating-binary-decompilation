\chapter{Scalable Validation of Binary Lifters}\label{chap:scalable-val}

Validating the correctness of binary lifters is pivotal to gain trust in
binary analysis, especially when used in scenarios where correctness is
important, e.g., in security analysis, binary patching, or recompilation
to other ISAs. Existing approaches focus on validating the correctness of
lifting a single instruction and do not scale to full programs. In this
work we show that formal translation validation of single instructions for
x86-64 is practical and develop a novel technique that uses validated
instructions to scale to program level validation, thereby eliminating the
bottleneck of heavy-weight equivalence checkers at the program level. Our
work is the first to to do \tv of single instructions on an architecture as 
extensive as \ISA,
uses the most precise formal semantics available, and has the widest
coverage in terms of number of instructions tested for correctness. 
%
To scale to whole programs, our \emph{compositional lifter} composes the validated IR
sequences to create a reference standard. The semantic equivalence check between the
reference and the lifter output is then reduced to a syntactic equivalence check through
use of a normalizer --- a procedure that reduces IR sequences to a
canonical representation.
%    
Using our approach, we find \sivFail new bugs in McSema, a mature open-source
lifter from x86-64 to LLVM IR. For whole programs, our approach was able to
prove equivalence of lifted code for \plvP/\plvT functions taken
from single-source benchmark test-suite.

    \section{Introduction}
\label{sec:Intro}

%% Why X86 is important?
% \ISA is undoubtedly the most widely used instruction set architecture on
% servers and personal computers which have grown to a remarkable complexity over
% the past few decades. 
%% Why analysis on X86 is important?
% Because of its wide-spread use, tools related to analysis and reasoning on the
% binary code are pervasive in software engineering and security research
% ~\cite{X}. Even there are situations where binary analysis seems more desirable than that
% on the source code. Examples are Commercial Off-The-Shelf software,
%   legacy code, or malware where the source code is not available. Even when
%   the source code is available, we cannot trust it when the compiler
%   generating the binary is not in the trusted computing base~\cite{Thompson}.
%   Lastly, even a trusted compiler can produce code which is semantically
%   different from the binary~\cite{WYSINWYE} and hence the desire to analyze the
%   binary itself.

The \ISA instruction set architecture (ISA) is one of the most complex and 
widely used ISAs on servers and desktops
which have grown to a remarkable complexity over the past few decades, 
and ensuring the correctness of the \ISA binary code is important.
%
The ability to directly reason about the binary code is desirable, not only because it allows to analyze the binary even when the source code is not available (e.g., legacy code or malware), but also because it avoids the need to trust the correctness of compilers~\cite{Thompson,WYSINWYE}.

% %
% Indeed, there exist various binary analysis tools, including those for software emulation and virtualization~\cite{QEMU:USENIX05,Valgrind:ENTCS03,DynamoRIO:2004,Pin:2005},
% malware analysis~\cite{BitBlaze:2008,BAP:CAV11,Egele:USENIX07,Yin:CCS07},
% reverse engineering~\cite{McSema:Recon14,Angr,Radare2}
% and sand-boxing~\cite{Kiriansky:2002:SEV,Erlingsson:2006,Yee:2009}.
% %
% These tools, however, are not designed to formally reason about the binary 

% depend, either explicitly or implicitly, on
% correct modeling of
%  the semantics of x86-64 instructions. 

A formal semantics of \ISA is required for formal reasoning about binary code, one of the strongest ways to ensure its correctness.
%
An \emph{executable} semantics is especially powerful because it allows direct testing to gain confidence in the definitions of the semantics, and also because it can allow additional tools based on symbolic execution, like deductive verification and symbolic test generation.
%
Completely formalizing the semantics of \ISA, however, is challenging especially due to the complexity and the sheer number of instructions that are informally specified in approximately 3,000-page standard~\cite{IntelManual}.
%

\paragraph{Existing Semantics for \ISA}

To date, to the best of our knowledge, despite several explicit attempts~\cite{Heule2016a,Goel:FMCAD14,Goel:ProCoS17} and other related systems~\cite{Leroy:2009,Remill,TSL:TOPLAS13,Hasabnis:ASPLOS16,Hasabnis:FSE16},
no \emph{complete} formal semantics of \ISA exists that can be used for formal reasoning about x86 binary programs.
%


Heule~\etal \cite{Heule2016a} presented a formal semantics of \ISA, but it covers only a fragment ($\sim$\strataPerc{}) of all instructions; as the authors of \cite{Heule2016a} candidly admitted, their synthesis methodology proved insufficient to add the remaining instructions primarily due to limitations of the underlying synthesis engine. 
%
Moreover, their semantics misses certain essential details (Section~\ref{sec:Approach} \& \ref{sec:Eval}).
% and it has not been demonstrated how to use their semantics to formally reason about the functional correctness of the \ISA binary.\footnote{Although they provide a tool that can extract SMT formulae describing each instruction's behavior from their semantics, we found errors in their tool and thus the generated SMT formulae. Also, most of the SMT solvers are not designed to support rich reasoning principles.}
% natively support all the reasoning principles of the full-fledged proof assistants.}
% for the general purpose formal reasoning.}
%
Also, it is not clear how to use their instruction-level semantics to the 
full-fledged 
theorem 
prover to be able to reason about the full functional correctness of the \ISA 
binary.
% Although it can provide the semantics in the form of the SMT formulae, the SMT solver 
%

Goel~\etal~\cite{Goel:FMCAD14,Goel:ProCoS17}, on the other hand, specified a formal semantics in the \SC{ACL2} proof assistant~\cite{ACL2:Kaufmann2000}, allowing to reason about functional correctness, but their semantics covers only a small fragment ($\sim$\goelPerc{}) of all user-level instructions.
%\Comment{Section 2 says they had only 191 unique opcodes, which is only about 20\%, and even less if you include system instructions. Which is correct?}
%

There also have been several attempts~\cite{Angr1,BAP:CAV11,Radare2,Hasabnis:FSE16} to \emph{indirectly} describe the \ISA semantics, where they define an intermediate language (IL), specify the IL semantics, and translate \ISA to the IL.
This indirect semantics, however, may not be general enough to be used for different types of formal analyses, since the IL might be designed with specific purposes in mind, not to mention that the translation may miss certain important details of the instruction behaviors.
%
Refer to Section~\ref{sec:RW} for a more detailed comparison to existing semantics.

\paragraph{Our Approach}

We present the most complete and thoroughly tested formal semantics of user-level \SC{\ISA assembly instructions}\footnote{\SC{The current work do not include a formal model of the binary instruction decoder. Note that, all future references of \ISA{} ``program(s)'' or ``instructions(s)'', in the context of our model, are meant to refer to the ``assembly language programs(s)'' or ``assembly instruction(s)''.}} to date.
We employed the \K framework~\cite{k-primer-2013-v32} (Section ~\ref{sec:KF}) as our formalism medium to leverage its capability of deriving various correct-by-construction formal analysis tools directly from the language semantics.
We took Heule~\etal~\cite{Heule2016a}'s semantics (Section ~\ref{sec:prelimstrata}) as our starting point to avoid duplicating the formalization effort. % made by the formal semantics research community.
We made several corrections or improvements to this semantics, to improve both soundness and efficiency.
We \emph{automatically} translated their semantics into \K, and cross-checked the translated semantics against the original using an SMT solver.
% cross-checked it by comparing the SMT formulae generated by each formalism, increasing our convince of the faithfulness of the translation.
We \emph{manually} specified the semantics of the remaining instructions 
faithfully consulting the Intel manual~\cite{IntelManual} to obtain the 
complete semantics. A manual specification \SC{may sound like a daunting 
effort} at first, but the fact that (1) \ISA is largely stable and changes 
slowly over time, and (2) the state-of-the-art synthesis techniques for 
language semantics (notably, \Strata~\cite{Heule2016a} and Hasabnis 
\etal~\cite{Hasabnis:ASPLOS16, Hasabnis:FSE16}) suffer from scalability and/or 
faithfulness issues (see Section~\ref{sec:Approach:Overview} and \ref{sec:RW} 
for details), makes the effort worth undertaking. Moreover, an important 
message of this work is that complete formal semantics of x86 is possible, and 
that is not only useful in itself but also to generate formal analysis tools.

%($\sim$40\%), obtaining the complete semantics.
Like closely related previous work~\cite{Goel:FMCAD14,Heule2016a}, we omit the relaxed memory model of \ISA and thus the concurrency-related operations.
Modelling concurrency is a complex but relatively orthogonal problem in the presence of small-step operational semantics, as shown in prior work~\cite{Sarkar:POPL09,Owens:x86-TSO}, where they have integrated their memory model with a small subset of $32$-bit x86 instruction set.
We believe that integrating such a memory model into our instruction semantics is a promising direction toward rigorously reasoning about real-world programs running on modern multiprocessors. We leave it for future work.


\paragraph{Contributions}

In a nutshell, below are our primary contributions towards defining the   
formal semantics of \ISA.

%DSAND: Include Intel counts
\emph{Completeness.~}
We present the most complete formal semantics of \ISA to date.
\SC{Specifically, our semantics formalizes all the user-level instructions of the \ISA Haswell ISA (that is,
\currentIS{} instructions covering \currentIntel{} mnemonics~\cite{IntelManual}), except deprecated ones (\Xmmx{} instructions),
the AES cryptography extensions (\crypto{} instructions), and the system \& concurrency-specific instructions (\system{} instructions) (Section~\ref{sec:IC})}.

\emph{Faithfulness.~}
Being executable, the semantics of \SC{\emph{each}} instruction has been thoroughly tested against 7,000+ test cases using the co-simulation method (Section~\ref{sec:Eval}).
We found errors in both the \ISA standard document and other existing semantics including the baseline semantics (Section~\ref{sec:Eval}).
%\SC{Note that, the testing of floating-point instructions is facilitated by the fact K already has matured library support for floating-point theories which we augmented to support modeling such instructions. In Section~\ref{sec:limit}, we reported a precision issue with our floating-point library support.}

\emph{Usability \& portability.~}
\AEC{We illustrate the potential of our semantics to be used for formal analyses such as deductive program verification and program equivalence checking (Section~\ref{sec:Appl}).}
The \K framework also enables one to represent our semantics as SMT theories,
% that can be handled by various SMT solvers,
which allows others to easily reuse our semantics for their own purposes.
\SC{%
Indeed, we have translated our semantics to Stoke~\cite{completing-stock} which can serve as a drop-in replacement of Heule~\etal's semantics~\cite{Heule2016a} and can immediately benefit tools built on Stoke (e.g., \cite{Roessle:CPP19}).
}
\cmt{

\SC{%
\emph{Semantics development practice.~}
Reflecting our \ISA semantics development effort, we identify certain important aspects to be considered when specifying a large instruction set architecture semantics, which we believe can be also applied to other large language semantics to a certain extent (Section~\ref{sec:lesson-learned}).
}}
%
% \paragraph{Artifacts}

Our formal semantics is publicly available at~\cite{x86-64-github}.


% In this paper, we present the most complete and thoroughly tested formal semantics of \ISA to date.
% Specifically, our semantics faithfully formalizes all the user-level instructions (\currentIS{} in total) of the \ISA Haswell ISA~\cite{IntelManual}.
% Our semantics is specified in the K framework~\cite{k-primer-2013-v32} that allows us to execute our semantics, and derive various correct-by-construction formal analysis tools directly from the semantics.
% Being executable, our semantics has been thoroughly tested against 7,000+ test cases using the co-simulation method (Section~\ref{sec:Eval}).
% We also demonstrate that our semantics can be used for various formal analyses, such as symbolic execution, deductive program verification, and program equivalence checking (Section~\ref{sec:Appl}).
% The K framework also allows us to represent our semantics in the SMT theories as well,
% % that can be handled by various SMT solvers,
% which allows others to easily reuse our semantics for their own purposes.




\subsection{Challenges in Formalizing \ISA}
\label{sec:challenges-in-formalizing-x86}

% In addition to the sheer number of instructions to be specified,
% % and the ambiguity of the informal standard document to consult,
% the following aspects make it challenging to completely specify the formal semantics of \ISA.

\paragraph{Size and Complexity}

\SC{The \ISA ISA has a large number of instructions, partly because of a large number of complex instructions and partly because it keeps most of the legacy and deprecated instructions ($\sim$ \Xmmx{}+) for the sake of backwards compatibility.}
It consists of \totalIntel{} mnemonics, and each mnemonic admits several variants, depending on the types (i.e., register, memory, or constant) and the size (i.e., the bit-width) of operands.

%\vspace{-2pt}
\paragraph{Inconsistent Instruction Variants}

Some variants have divergent behaviors more than the difference of their type and size. For example, \instr{movsd}, one of the 128-bit SSE instructions, has very different behaviors depending on whether the type of the source operand is register or memory; it clears the higher 64 bits of the target register only when the source type is memory.
Indeed, we revealed bugs in other semantics due to their incorrect generalization of the variants' behavior (Details in Section ~\ref{sec:Approach}, Instruction Variants).

%\vspace{-2pt}
\paragraph{Ambiguous Documentation}

The \ISA reference manual informally explains the instruction behaviors, leaving certain details unspecified or ambiguous, which required us to consult with an actual processor implementation to clarify such details.
%
Completely formalizing the vast number of instructions with carefully identifying all the corner cases from the informal document, thus, is highly non-trivial.
% is a huge effort that may not seem to be feasible.
% , simply because of the sheer number of instructions to be formalized.
% \ISA has the special register \reg{rflags} that stores the current state of the processor.

%\vspace{-2pt}
\paragraph{Undefined Behaviors}

The \ISA standard also admits undefined behaviors that are implementation-dependent.
Many instructions (\undefIntel{}\footnote{\label{note1}These numbers are obtained by parsing the official manual ``Volume 2: Instruction Set Reference'' and cross checked with projects~\cite{Stoke2013, Felix} investing similar efforts.} out of \totalIntel{} mnemonics) have undefined behaviors: their output values of the destination register or the \reg{rflags} register are undefined in certain cases.
% behaviors, for which each processor can choose any behavior.
% For example, the bit-scan-forward instruction \instr{bsf} that computes the bit index of the least significant set bit in the source operand is not defined when the source operand's value is 0.
That is, the processor is free to choose any behavior in undefined cases.
% (i.e., no bit 1 appears), however, its output is implementation-dependent.
% More than \undefPerc{}\% of all instructions admit undefined behaviors!
%

Many existing semantics, however, simply ``define'' the undefined behaviors by
% consulting with an actual processor implementation.
following a specific behavior taken by a processor implementation.
This approach is problematic because they do not capture all possible behaviors of different processor implementations.
Indeed, we found discrepancies between existing semantics in specifying the undefined behaviors, where different semantics are valid only for different groups of processors.
That is, such semantics are not adequate to formally reason about universal properties (e.g., portability) of a program that need to be satisfied for all standard-conforming processors.
%
For example, the parity flag \reg{pf} is undefined in the logical-and-not instruction \instr{andn}, where the processor implementation is allowed to either update the flag value (to 0 or 1), or keep the previous value.
We found, e.g., that Remill~\cite{Remill} updates the flag with 0, whereas Radare~\cite{Radare2} keeps it unmodified.
% We found that Strata~\cite{Heule2016a} updates the flag based on the result of the \instr{andn} operation, whereas a binary analysis tool Radare~\cite{Radare2}, for example, keeps it unmodified.
%
Identifying and faithfully specifying all of the undefined behaviors, thus, are desirable but challenging.

\SC{In our semantics, we faithfully modeled the undefined value as a unique symbol (called \s{undef}) whose value is nondeterministically decided each time within the proper range.}
These nondeterministic values are enough to capture and formally reason about 
all possible behaviors of the instructions for different processors (and even 
any future, standard-conforming processor).
While performing instruction-level testing (Section~\ref{sec:Eval}), we 
consider the \s{undef} symbol to be matched with any concrete value provided by 
the hardware, so that we can test the instructions modulo the undefined 
behaviors.




% \paragraph{Undefined, Implementation-Dependent Behaviors}

% According to the \ISA standard, 

% We found that other semantics do not faithfully model the undefined behaviors, simply following a specific behavior taken by a processor implementation.
%









% many of the existing semantics do not faithfully specify the implementation-dependent semantics, and divergent behaviors across the different semantics.

% Naively specifying the implementation-dependent behaviors by following the behaviors of an existing processor is problematic, because it cannot capture all possible behaviors of different processors.\footnote{Even if all of the existing processors agree on a certain behavior, it may not be the case in the future processors.}
% That is, a program verified w.r.t. such a naive semantics may have different behaviors in another processor (in the future).
% Identifying and faithfully specifying all of the implementation-dependent behaviors are challenging.
% Indeed, we found that many of the existing semantics do not faithfully specify the implementation-dependent semantics, and divergent behaviors across the different semantics. % , which may not be sound for different processors.

% A naive approach of specifying the implementation-dependent behaviors would be to follow the behaviors of an existing processor.
% This approach has the benefits of having the co-simulation based testing straightforward.
% However, the naively defined semantics cannot capture all possible behaviors of different processors.\footnote{Even if all of the existing processors agree on a certain behavior, it may not be the case in the future processors.}
% That is, a program verified w.r.t. such a naive semantics may have different behaviors in another processor (in the future).


% In effect, these
% approaches restricts the value of the flag to a concrete value and hence a
% semantics model based on these approaches prevent exploration of paths feasible
% in some processor implementation.  Therefore, it is desirable to identify these
% cases where a register or a flag could be undefined and to encode this
% information in the model in such a way so as to assist exploration of all
% resulting execution paths.






%  However, such a modeling is difficult to obtain given the fact that the ISA is
%  overly complex and the only published description of the \ISA ISA is the Intel
%  manual~\cite{IntelManual} with over 3000 pages written in an ad-hoc combination of
%  English and pseudo-code. Also Intel does not appear to have a formal model (not even internally) that fully defines CPU behavior (~\cite{Amit:SOSP15}, Section 4.1). This informal nature of the reference specification
%  imposes a challenge in ensuring correctness of the developed formal 
%  semantics.
%  Most of the existing binary analysis tools mitigate the challenge by manually modeling  the specification of the semantics in their analysis IR which are then validated against the actual hardware to attain faithfulness of the model.

% %% What are we offering?
% Our goal is to formally model the semantics of all the user-level \ISA Haswell ISA, which is used widely for server and desktop machines nowadays. This work will not only help in formal reasoning on binary programs but also serve as a comparison reference for the existing binary analysis projects. In this paper we mention the challenges we faced and the lessons we learned while doing so. 

% \subsection{Why Yet Another \ISA Semantics}
% A formal semantics of a language is foundational to any formal reasoning about programs written in that language and serves as a trust base on which the faithfulness of downstream analyses and reasoning tools depends. Following are some of the desirable properties of a formal model:
% \begin{itemize}
%     \item \textbf{Completeness: }To make the model applicable to real world programs.
%     \item \textbf{Executable: }To compare the model against a reference implementation, which in turn enhances the faithfulness of the model.     
%     \item \textbf{Applicable for program reasoning \& verification: }To reason about arbitrary high-level properties on the program using a general reasoning infrastructure such as theorem prover.
%     \item \textbf{Same artifact being used for both execution and formal reasoning: } To make the trust base smaller by having a single artifact used for both execution and formal reasoning. This also adds to the faithfulness of the model. Also having separate formalizations incur the overhead of maintaining both.
%     \revisit{an example to show the relevance of this point}
% \end{itemize}

% %\revisit{Can we say McSema has formal spec? as I am not sure if McSema llvm based spec is to be called formal}
% Several research groups have implemented, with considerable effort, their own
%  instruction semantics specification for the \ISA instruction set. Notables are \Strata~\cite{Heule2016a} and Goel et al.~\cite{Goel:FMCAD14, Goel:ProCoS17, Goel:VTTE13:ACP:2958657.2958669} which are considered state of the art in defining the formal semantics. 
%  \Strata provides fairly complete support of instruction semantics but we found that the semantics specifications are not ready to be used directly for a solid foundation of formal reasoning. Indeed, we found several issues and inconsistencies in their semantics (more details in section~\ref{sec:Eval}).  Also, it is not clear how to connect the \Strata semantics to a full-fledged theorem prover in order to reason about the full functional correctness of the \ISA binary programs. On the other hand, the Goel et al. semantics is far from being complete w.r.t user-level instruction coverage ($25.54\%$). As shown in Table~\ref{table:RW}, no existing semantics meet all the desired properties mentioned above (details about the comparison are mentioned in section~\ref{sec:RW}). Hence we developed a formal \ISA semantics
% in order to have a single, clean-slate semantics that can be used not
% only as a reference model for \ISA ISA, but also to develop formal
% analysis tools for it.






% \subsection{Our Appraoch}\label{sec:M} 
% %% What is the methodology
% Because of the huge volume of the user-level instructions available in Intel
% Manual, we avoided modeling the entire set from scratch and decided to reuse
% the effort and research already invested by other projects. Towards that goal,
%     we chose to borrow the instruction semantics from project
%     \Strata~\cite{Heule2016a}, which supports \strataPerc{} of the total user-level
%     instructions. We modeled the semantics of remaining \currentManualPerc{} instructions by
%     carefully consulting the Intel manual and put significant effort in
%     validating the semantics (as detailed in section~\ref{sec:Eval}). More details about this
%     aspect of our approach is presented in section~\ref{sec:harvestsema}. The
%     choice of \Strata is based on the fact that 1. It has fairly complete
%     instruction support, and 2. The semantic specifications are better suited,
%     to be used as a starting point  towards defining formal semantics, than
%     other specifications used in related projects having larger instruction
%     support (refer to section~\ref{sec:RW} for more details). 

% As detailed in section~\ref{sec:x86sema}, we employed \K~\cite{k-primer-2013-v32} to define the formal semantics of \ISA
% ISA. Given that semantics, \K provides, at no additional cost,  an execution
% engine, which yields an interpreter for the defined language, as well as a
% sound and relatively complete deductive verification system based on symbolic
% execution, which can be used to reason about \ISA programs.  \cmt{With that our
%   x86 model serves both as an executable instruction-set simulator and a formal
%     specification that is used to proof high-level properties about machine
%     code.} The choice of \K is based on the additional fact that it is highly
%     specialized in defining language semantics (as demonstrated by the full
%         formal semantics of production languages like C and
%         LLVM~\cite{Ellison,KC, KLLVM}) and  has high level of automation and
%     expressive power\Qd{should I add "than other contemporary language
%       formalization framework". Also  let me know if there is a better way to
%         put the last sentence}.  


 
% %%%%%%%%%%%%%%%%%%%%%%%%%%%%%%%%%%%%%%%%%%%%%%%%%%%%%%%%%%%%%%%%%%%%%%
% %%%%%%%%%%%%%%%%%%%%%%%%%%%%%%%%%%%%%%%%%%%%%%%%%%%%%%%%%%%%%%%%%%%%%% 
% \subsection{Contribution}
% Our contributions can be listed as follows:
% \begin{itemize}
%     \item Developed a complete formal semantics of the \ISA user-level ISA. 
%     \item Thoroughly tested the instructions semantics using
%     \begin{itemize}
%         \item Instruction level testing
%         \item Program level testing
%     \end{itemize}
%     While doing so, we identify several important errors in pre-existing formalizations including Intel Manuals. Moreover, our specification of the x86 ISA is regularly validated to increase faithfulness of the semantics and the trust in the applicability of the results of formal analysis.
%     \item Applied the semantics to formal reasoning on \ISA programs.
%     \begin{itemize}
%         \item Verified functional correctness of toy \Qd{I think non-trivial is a better choice than toy} programs, like \emph{sum-to-n}.
%         \item Generated test cases using symbolic execution, for detecting security vulnerability.
%         \item Checked equivalences between \ISA programs across different optimizations.
%         \item Verified the translation of binary lifters targeting LLVM IR~\cite{McSema:Recon14, FCD}.
%         \revisit{Need to reorder the sequence  once we have the x86 -> llvm validator in place} 
%     \end{itemize}
% \end{itemize}

% \subsection{Outline}
% Next, we present our formal, executable model of the x86 ISA from an
% engineering standpoint and describe our design decisions and challenges in
% extending Strata in detail.

\section{Approach Overview}
\label{sec:Approach}

\makeatletter
\tikzset{
    database/.style={
        path picture={
            \draw (0, 1.5*\database@segmentheight) circle [x radius=\database@radius,y radius=\database@aspectratio*\database@radius];
            \draw (-\database@radius, 0.5*\database@segmentheight) arc [start angle=180,end angle=360,x radius=\database@radius, y radius=\database@aspectratio*\database@radius];
            \draw (-\database@radius,-0.5*\database@segmentheight) arc [start angle=180,end angle=360,x radius=\database@radius, y radius=\database@aspectratio*\database@radius];
            \draw (-\database@radius,1.5*\database@segmentheight) -- ++(0,-3*\database@segmentheight) arc [start angle=180,end angle=360,x radius=\database@radius, y radius=\database@aspectratio*\database@radius] -- ++(0,3*\database@segmentheight);
        },
        minimum width=2*\database@radius + \pgflinewidth,
        minimum height=3*\database@segmentheight + 2*\database@aspectratio*\database@radius + \pgflinewidth,
    },
    database segment height/.store in=\database@segmentheight,
    database radius/.store in=\database@radius,
    database aspect ratio/.store in=\database@aspectratio,
    database segment height=0.1cm,
    database radius=0.25cm,
    database aspect ratio=0.35,
}
\makeatother

\begin{figure*}
\begin{tikzpicture}[%node distance=2.5cm,
every node/.style={fill=white}, align=center]
\tikzset{%
    %>={Latex[width=2mm,length=2mm]},
    % Specifications for style of nodes:
    base/.style = {rectangle, rounded corners,
         text centered},
    process/.style = {base,  fill=orange!15, font=\ttfamily, draw=black},
    basic box/.style = {
        shape = rectangle,
        align = center,
        draw  = #1,
        %fill  = #1!25,
        rounded corners},
      header node/.style = {
        %Minimum Width = header nodes,
        font          = \strut\Large\ttfamily,
        text depth    = +0pt,
        fill          = white,
        draw},
      header/.style = {%
        inner ysep = +1.5em,
        append after command = {
            \pgfextra{\let\TikZlastnode\tikzlastnode}
            node [header node] (header-\TikZlastnode) at (\TikZlastnode.north) {#1}
            %node [span = (\TikZlastnode)(header-\TikZlastnode)] at (fit bounding box) (h-\TikZlastnode) {}
        }
      },
}
\def\blockhdist{2cm}
\def\blockvdist{1.5cm}
\def\phasehdist{8cm}


%%%%%%%%%%%%%%% PHASE I
\node (instr)          [base]  {\ISA\\ Instruction, $I$};
\node (SIVmcsema)   [process, below of=instr, yshift=-0.5cm, xshift=\blockhdist]  {Decompiler,$D$\\(unter test)};
\node (irseq)          [base,below of=SIVmcsema, xshift=\blockvdist]  {IR Sequence, $L$};
\node (instrSymEx)   [process, below of=instr, yshift=-2*\blockvdist, xshift=-\blockhdist] {Symbolic Ex.\\(\ISA semantics) };
\node (irSymEx)   [process, below of=irseq, yshift=-0.5cm] {Symbolic Ex.\\IR semantics};
\node (proofGen)   [process, below of=instr, yshift=-4*\blockvdist] {Proof Script Generator};
\node (solver)   [process, below of=proofGen, yshift=-\blockvdist] {Z3 Solver};
\node (decide1)     [draw, below of=solver, yshift=-1cm, diamond, aspect=2]  {$R == unsat$};
\node (report1)       [left of=decide1, xshift=-\phasehdist/4]  {Report};
\node (caption1)     [below of=solver, yshift=-2.5cm] {(a) Instruction Level validation};

\draw[->]             (instr) -- (SIVmcsema);
\draw[->]             (SIVmcsema) -- (irseq);
\draw[->]     (instr) -- (instrSymEx);
\draw[->]     (irseq) -- (irSymEx);
\draw[->]     (instrSymEx) -- node {Symbolic\\Summary, $sum_{\ISA}$} (proofGen);
\draw[->]     (irSymEx) -- node {Summary\\$sum_{ir}$ } (proofGen);
\draw[->]     (proofGen) -- node {Proof Script\\$z3.solve(sum_{\ISA} \ne sum_{ir})$} (solver);
\draw[->]     (solver.south) -- node {$R$} (decide1.north);
\draw[->]     (decide1.west) -- node {no} (report1);

%%%%%% Store
\node (store)     [process, right of=solver, xshift=\phasehdist/2]{Instructions' Dababase\\Store({$I$,$L$})};
\draw[->]       (decide1.east) -| node[xshift=-2cm] {$yes$} (store.south);


%%%%%%%%%%%% PHASE II
\node (start)             [base,right of=instr, xshift=\phasehdist]                       {\ISA Program};
\node (compd)             [process, below of=start, yshift=-0.5cm, xshift=-\blockhdist]          {Compositional\\Decompiler};
\node (mcsema)             [process, below of=start, yshift=-0.5cm, xshift=\blockhdist]          {Decompiler,$D$\\(under test)};
\node (normalizer1)             [process, below of=compd, yshift=-\blockvdist]   {Normalizer};
\node (normalizer2)         [process, below of=mcsema, yshift=-\blockvdist]   {Normalizer};
\node (matcher)     [process, below of=normalizer2, yshift=-\blockvdist, xshift=-\blockhdist]   {Syntax Based Matcher};
\node (decide2)     at (decide1 -| matcher) [draw,  diamond, aspect=2]  {$M == equiv$};
\node (caption2)     at (caption1 -| matcher) {(b) Program Level validation};
\node (report2)       [left of=decide2, xshift=-\phasehdist/4]  {Report};
 
 
\draw[->]             (start) -- (compd);
\draw[->]             (start) -- (mcsema);
\draw[->]     (compd) -- node {Proposed IR} (normalizer1);
\draw[->]     (mcsema) -- node {Lifted IR} (normalizer2);
\draw[->]     (normalizer1) -- node {normalized\\IR} (matcher);
\draw[->]     (normalizer2) -- node {normalized\\IR} (matcher);
\draw[->]     (matcher) -- node {Matcher Results, $M$} (decide2);
\draw[->]     (store.north) |-  (compd.west);
\draw[->]     (decide2.west) -- node {no} (report2);

%% Outter box
\begin{scope}[on background layer]
\node[fit = (compd)(mcsema)(start)(matcher), basic box = black,] (Phase1) {};
    \node[fit = (compd)(mcsema)(start)(matcher)(instr)(solver)(irseq)(instrSymEx)(irSymEx)(caption1)(caption2), basic box = black,] (Overview) {};
\end{scope}

\end{tikzpicture}


\end{figure*}



\section{Single \ISA Instruction Validation}
\section{Preliminaries}
\label{sec:Prelim}

Here we provide background on the \K framework and the \Strata project~\cite{Heule2016a} (used for our baseline semantics).

% briefly explain pieces of \ISA ISA necessary for our
% presentation. We also talk about \K, a semantics engineering tool which we
% chose to formalize our semantics into and  \Strata which gives us a head-start towards modeling the semantics of \ISA instructions.

%%% (VSA) DISABLE X86 BACKGROUND: IT IS WIDELY KNOWN
%\subsection{\ISA Instruction Set Architecture}

\ISA is the 64-bit extension of x86, a family of backward-compatible ISAs.
% x64 is a generic name for the 64-bit extensions to Intel's and AMD's 32-bit x86 instruction set architecture (ISA). 
%
We briefly explain some details of the architecture.

% Registers:
\ISA has the sixteen 64-bit general purpose registers (\reg{rax}--\reg{rdx}, \reg{rsi}, \reg{rdi}, \reg{rsp}, \reg{rbp}, \reg{r8}--\reg{r15}), and the two $64$-bit special registers (\reg{rip} and \reg{rflags}).
The lower $32$-, $16$- and $8$-bit portions of the register are referenced by the sub-register variants, e.g., \reg{eax}, \reg{ax}, and \reg{al} for \reg{rax}, respectively.
% \cmt{All these registers are used for storing integer values.}
The Haswell \ISA ISA additionally has sixteen 256-bit SIMD registers (\reg{ymm0}--\reg{ymm15}) along with the lower 128-bit sub-register variants (\reg{xmm0}--\reg{xmm15}).
% \cmt{, which are used for floating-point and packed operations.}

The \reg{rflags} register stores the current state of the processor.
Specifically, for example, the status flags such as the carry flag (\s{cf}), the parity flag (\s{pf}), the adjust flag (\s{af}), the zero flag (\s{zf}), and the sign flag (\s{sf}) are stored in \reg{rflags}.
These status flags are set according to the result of arithmetic and logical instructions.
% the status flags used mostly in user-level \ISA programs, and updated by arithmetic-logical instructions according to the result of the operation.
Some of control transfer instructions are performed based on the values of these flags.

% Addressing modes: 
\ISA supports the addressing modes, expressions that calculate a memory address to be read or written to. The addressing modes are used as the source or the destination of instructions that access the memory.
The addressing mode expressions can be generalized as:
$\s{base} + \s{index} \times \s{stride} + \s{offset}$.
In the assembly code, for example, \instr{-4(\reg{rax}, \reg{rbx}, 8)} denotes the address mode ``$\reg{rax} + \reg{rbx} \times 8 - 4$''.

% \cmt{      
% \begin{lstlisting}[style=SIMPRULES]
%     D (%RB, %RI, S)
%         RB is register for base
%         RI is register for index (0 if empty)
%         D is displacement (0 if empty)
%         S is scale 1, 2, 4 or 8 (1 if empty)
%     Effective Address: Mem[%RB + D + S*%RI]
% \end{lstlisting}
% }   

% Instruction variants:
% If an instruction access memory, we call it memory instruction. Else, if the instruction takes constant, then it is called immediate instruction. Otherwise, it is a register instructions.
\ISA has three types of instructions depending on the types of their operands: register instructions (with only register operands), memory instructions (with address mode operands), and immediate instructions (with constant operands).
The same mnemonic can be used for the different types of instructions.
For example, the mnemonic \instr{add} can be used for the register instructions (e.g., \instr{addq \reg{rax}, \reg{rbx}}\footnote{Throughout the presentation we will be using the {AT\&T} assembly syntax~\cite{Syntax} where the destination operand comes after source operands.}), the memory instructions (e.g., \instr{addq -4(\reg{rax}), \reg{rbx}}), and the immediate instructions (e.g., \instr{addq \$1, \reg{rbx}}).
% Throughout this paper, we will count each type of instructions as a different instruction.
% In the presentation, we will count them as different insructions ( and call them  instructions variants). Also different opcodes in the Intel manual are counted separately towards total instruction count. 
% \cmt{Whereas, register instructions differing on register names, immediate instructions differing on  constant values and memory instructions   differing on  memory addresses will be counted once towards total instruction count. }




\subsection{K Framework}\label{sec:KF}

%\K~\cite{k-primer-2013-v32} \cmt{\url{http://kframework.org}} is a framework for
%defining formal language semantics. Given a syntax and a semantics of a language, \K
%generates a parser, an interpreter, as well as formal analysis tools such as
%model checkers and deductive program verifiers, at no additional cost. Using
%the interpreter, one can test their semantics immediately, which significantly
%increases the efficiency of semantics developments. Furthermore, the formal
%analysis tools facilitate formal reasoning about the given language semantics.
%This helps both in terms of applicability of the semantics and in terms of
%engineering the semantics itself.
%
%We   refer the reader to~\cite{k-primer-2013-v32, rosu-serbanuta-2010-jlap} for
% details. In a nutshell, in \K, a language syntax is given using conventional
%Backus-Naur Form (BNF). A language semantics is given as a parametric
%transition system, specifically a set of reduction rules over configurations. A configuration is an algebraic representation of the
%program code and state. Intuitively, it is a tuple whose elements
%(called cells) are labeled and possibly nested. Each cell represents a
%semantic component, such as the memory or the registers. A special cell, named \s{k}, contains a
%list of computations to be executed. A computation is essentially
%a program fragment, while the original program is flattened into a
%sequence of computations. A rule describes a one-step transition
%between configurations, giving semantics to language
%constructs. Rules are modular; they mention only relevant cells that
%are needed in each rule, making many rules far more concise and easy to read
%than in some other formalisms.



% One of the most appealing aspects of K is its modularity. It is very rarely the
% case that one needs to touch existing rules in order to add a new feature to
% the language. This is achieved by structuring the configuration as nested cells
% and by requiring the language designer to mention only the cells that are
% needed in each rule, and only the needed portions of those cells. For example,
% the above rule only refers to the \s{k} and \s{regstate} cells, while
% the entire configuration contains many more cells (Figure
% ~\ref{fig:config}). This modularity makes for compact and human
% readable semantics, and also helps with the overall effectiveness of the
% semantics development. For example, even if new cells are later added to
% configuration, to support new features, the above rule does not change.

% \input{figures/configuration.tex}

\subsection{Strata Project}\label{sec:prelimstrata}
%% Support Count 
\Strata~\cite{Heule2016a} automatically synthesized formal semantics of \strataWithDupIS{} instruction variants (representing \strataIntel{} unique mnemonics) of the \ISA Haswell ISA. 
%% Set of test input 
The algorithm to learn the formal semantics of an instruction, say \s{IS}, starts with a small set of instructions, called base set \s{B}, whose semantics are known and trusted; a set of test inputs \s{T}, and the output behavior of \s{IS} obtained by executing \s{IS} on \s{T}. Then \Stoke~\cite{Stoke2013} is used  to synthesize  instruction sequences which contain instructions from \s{B} and match the behavior of \s{IS} for all test cases in \s{T}. Given two such generated instruction sequences \s{IS} and \s{IS}$^\prime$, their equivalence is decided  using an SMT solver and the trusted and known  semantics from the base set. If the two sequences are
semantically distinct, then the model produced by the SMT solver is used  to obtain
an input \s{t} that distinguishes \s{IS} and \s{IS}$^\prime$, and \s{t} is added to \s{T}. This process of synthesizing instruction sequence candidates and accepting or rejecting them based on equivalence checking with previous candidates, is repeated until a threshold is reached, which in their implementation is based on the number of accepted instruction sequences. 

%% Generalization 
\cmt{
Using the above technique, they first came up with the semantics of $692$ register
and $120$ immediate instructions. Then they used ``generalization'' of the register
instructions to get a total support count of \strataWithDupIS. Generalization is based on their hypothesis that the memory or immediate variants will behave identically with corresponding register variant (other than where the inputs come from) and hence can use the same formula as register variants. They validate this hypothesis using random testing. }

%% Two format of output
For each instruction, \Strata manifested its semantics in terms of two related artifacts.
The first artifact is an instruction sequence and the second is a set of SMT formulas 
in the bit-vector theory, one for each output register. 
The second is obtained by symbolically executing the first.

\cmt{
\Stoke~\cite{Stoke2013} contains manually written specifications  for a subset of the
x86-64 instruction set and there are APIs to generate SMT formulas from those specifications. 
\Strata uses those formulas to compare against the ones learned
by stratification by asking an SMT solver if the formulas are equivalent. For
example, for the instruction \instr{andnq \%rdx, \%rcx, \%rbx} the manually
written specification looks like the following:

\begin{lstlisting}[style=C++, label={lst:CS5}, caption={Manually Writen  specification for \instr{andnq
\%rdx, \%rcx, \%rbx}}]
semantic_function (
  Operand dst, Operand src1, Operand src2, 
  // dst == %rbx, src1 == %rcx, src2 == %rbx
  SymBitVector a, SymBitVector b, SymBitVector c, 
  // a, b, c are symbolic values for dst, src1, 
  // src2 resp. 
            SymState& ss) {
  // The symbolic state to be updated with the 
  // behaviour of the instruction     
  auto tmp = (!b) & c;
  
  // Setting destinaton
  ss.set(dst, tmp); 
  // Setting of, cf to false 
  ss.set(eflags_of, SymBool::_false()); 
  ss.set(eflags_cf, SymBool::_false());
  // Updating sf, zf based on result
  ss.set(eflags_sf, tmp[dst.size()-1]);
  ss.set(eflags_zf, tmp == SymBitVector::constant(dst.size(), 0));
  // Setting af, pf to undefined values
  ss.set(eflags_af, SymBool::tmp_var());
  ss.set(eflags_pf, SymBool::tmp_var());
}
\end{lstlisting}
}

\subsection{McSema}\label{sec:McS}



\section{\ISA Program-Level Validation}
\subsection{Compositional Decompiler}
\subsection{Normalizer}
\subsection{Matcher}

\section{Preliminaries}
\label{sec:Prelim}

Here we provide background on the \K framework and the \Strata project~\cite{Heule2016a} (used for our baseline semantics).

% briefly explain pieces of \ISA ISA necessary for our
% presentation. We also talk about \K, a semantics engineering tool which we
% chose to formalize our semantics into and  \Strata which gives us a head-start towards modeling the semantics of \ISA instructions.

%%% (VSA) DISABLE X86 BACKGROUND: IT IS WIDELY KNOWN
%\subsection{\ISA Instruction Set Architecture}

\ISA is the 64-bit extension of x86, a family of backward-compatible ISAs.
% x64 is a generic name for the 64-bit extensions to Intel's and AMD's 32-bit x86 instruction set architecture (ISA). 
%
We briefly explain some details of the architecture.

% Registers:
\ISA has the sixteen 64-bit general purpose registers (\reg{rax}--\reg{rdx}, \reg{rsi}, \reg{rdi}, \reg{rsp}, \reg{rbp}, \reg{r8}--\reg{r15}), and the two $64$-bit special registers (\reg{rip} and \reg{rflags}).
The lower $32$-, $16$- and $8$-bit portions of the register are referenced by the sub-register variants, e.g., \reg{eax}, \reg{ax}, and \reg{al} for \reg{rax}, respectively.
% \cmt{All these registers are used for storing integer values.}
The Haswell \ISA ISA additionally has sixteen 256-bit SIMD registers (\reg{ymm0}--\reg{ymm15}) along with the lower 128-bit sub-register variants (\reg{xmm0}--\reg{xmm15}).
% \cmt{, which are used for floating-point and packed operations.}

The \reg{rflags} register stores the current state of the processor.
Specifically, for example, the status flags such as the carry flag (\s{cf}), the parity flag (\s{pf}), the adjust flag (\s{af}), the zero flag (\s{zf}), and the sign flag (\s{sf}) are stored in \reg{rflags}.
These status flags are set according to the result of arithmetic and logical instructions.
% the status flags used mostly in user-level \ISA programs, and updated by arithmetic-logical instructions according to the result of the operation.
Some of control transfer instructions are performed based on the values of these flags.

% Addressing modes: 
\ISA supports the addressing modes, expressions that calculate a memory address to be read or written to. The addressing modes are used as the source or the destination of instructions that access the memory.
The addressing mode expressions can be generalized as:
$\s{base} + \s{index} \times \s{stride} + \s{offset}$.
In the assembly code, for example, \instr{-4(\reg{rax}, \reg{rbx}, 8)} denotes the address mode ``$\reg{rax} + \reg{rbx} \times 8 - 4$''.

% \cmt{      
% \begin{lstlisting}[style=SIMPRULES]
%     D (%RB, %RI, S)
%         RB is register for base
%         RI is register for index (0 if empty)
%         D is displacement (0 if empty)
%         S is scale 1, 2, 4 or 8 (1 if empty)
%     Effective Address: Mem[%RB + D + S*%RI]
% \end{lstlisting}
% }   

% Instruction variants:
% If an instruction access memory, we call it memory instruction. Else, if the instruction takes constant, then it is called immediate instruction. Otherwise, it is a register instructions.
\ISA has three types of instructions depending on the types of their operands: register instructions (with only register operands), memory instructions (with address mode operands), and immediate instructions (with constant operands).
The same mnemonic can be used for the different types of instructions.
For example, the mnemonic \instr{add} can be used for the register instructions (e.g., \instr{addq \reg{rax}, \reg{rbx}}\footnote{Throughout the presentation we will be using the {AT\&T} assembly syntax~\cite{Syntax} where the destination operand comes after source operands.}), the memory instructions (e.g., \instr{addq -4(\reg{rax}), \reg{rbx}}), and the immediate instructions (e.g., \instr{addq \$1, \reg{rbx}}).
% Throughout this paper, we will count each type of instructions as a different instruction.
% In the presentation, we will count them as different insructions ( and call them  instructions variants). Also different opcodes in the Intel manual are counted separately towards total instruction count. 
% \cmt{Whereas, register instructions differing on register names, immediate instructions differing on  constant values and memory instructions   differing on  memory addresses will be counted once towards total instruction count. }




\subsection{K Framework}\label{sec:KF}

%\K~\cite{k-primer-2013-v32} \cmt{\url{http://kframework.org}} is a framework for
%defining formal language semantics. Given a syntax and a semantics of a language, \K
%generates a parser, an interpreter, as well as formal analysis tools such as
%model checkers and deductive program verifiers, at no additional cost. Using
%the interpreter, one can test their semantics immediately, which significantly
%increases the efficiency of semantics developments. Furthermore, the formal
%analysis tools facilitate formal reasoning about the given language semantics.
%This helps both in terms of applicability of the semantics and in terms of
%engineering the semantics itself.
%
%We   refer the reader to~\cite{k-primer-2013-v32, rosu-serbanuta-2010-jlap} for
% details. In a nutshell, in \K, a language syntax is given using conventional
%Backus-Naur Form (BNF). A language semantics is given as a parametric
%transition system, specifically a set of reduction rules over configurations. A configuration is an algebraic representation of the
%program code and state. Intuitively, it is a tuple whose elements
%(called cells) are labeled and possibly nested. Each cell represents a
%semantic component, such as the memory or the registers. A special cell, named \s{k}, contains a
%list of computations to be executed. A computation is essentially
%a program fragment, while the original program is flattened into a
%sequence of computations. A rule describes a one-step transition
%between configurations, giving semantics to language
%constructs. Rules are modular; they mention only relevant cells that
%are needed in each rule, making many rules far more concise and easy to read
%than in some other formalisms.



% One of the most appealing aspects of K is its modularity. It is very rarely the
% case that one needs to touch existing rules in order to add a new feature to
% the language. This is achieved by structuring the configuration as nested cells
% and by requiring the language designer to mention only the cells that are
% needed in each rule, and only the needed portions of those cells. For example,
% the above rule only refers to the \s{k} and \s{regstate} cells, while
% the entire configuration contains many more cells (Figure
% ~\ref{fig:config}). This modularity makes for compact and human
% readable semantics, and also helps with the overall effectiveness of the
% semantics development. For example, even if new cells are later added to
% configuration, to support new features, the above rule does not change.

% \begin{figure}
  \centering
  %
  \renewcommand{\dotCt}[1]{\scriptstyle\s{#1}}
  \newcommand{\regstate}{\scriptstyle\s{ID}_\s{regname} \;\mapsto\; \s{Value}}
  \newcommand{\argc}{\scriptstyle\s{Int}}
  \newcommand{\argv}{\scriptstyle\s{Address}} 
  \newcommand{\entrypoint}{\scriptstyle\s{Address}}
%  \newcommand{\labeladdress}{\scriptstyle\textit{ID}_\textit{Label Name} \;\mapsto\; \textit{Address}}
  \newcommand{\codemem}{\scriptstyle\textit{Address} \;\mapsto\; \textit{Instruction}}
  \newcommand{\datamem}{\scriptstyle\s{Address} \;\mapsto\; \s{Value}}
  \newcommand{\bssmem}{\scriptstyle\textit{Address} \;\mapsto\; \textit{Value}}
  \newcommand{\stackmem}{\scriptstyle\textit{Address} \;\mapsto\; \textit{Value}}
  \newcommand{\heapmem}{\scriptstyle\textit{Address} \;\mapsto\; \textit{Value}}
  %
$
\kall{T}{
  \begin{array}{@{}c@{}}
  \kall{k}{\dotCt{K}} \mathrel{}
  \mathrel{}
  \kall{regstate}{\regstate} \mathrel{}
  \mathrel{}
  \kall{memstate} {\datamem} \mathrel{}
  \mathrel{}
  \cdots
  \cmt{ 
  \kall{executionEnv}{ 
  \kall{commandineArgs}{
    \kall{argc}{\argc} \mathrel{}    
    \kall{argv}{\argv} \mathrel{}
  } \mathrel{}
  \kall{entryPoint}{\entrypoint} \mathrel{}
}}
%  \kall{labelAddress}{\labeladdress} \mathrel{}
  \end{array}
}
$
  %
\vspace{-10pt}
  \caption{Program Configuration}
  \label{fig:config}
\end{figure}


\subsection{Strata Project}\label{sec:prelimstrata}
%% Support Count 
\Strata~\cite{Heule2016a} automatically synthesized formal semantics of \strataWithDupIS{} instruction variants (representing \strataIntel{} unique mnemonics) of the \ISA Haswell ISA. 
%% Set of test input 
The algorithm to learn the formal semantics of an instruction, say \s{IS}, starts with a small set of instructions, called base set \s{B}, whose semantics are known and trusted; a set of test inputs \s{T}, and the output behavior of \s{IS} obtained by executing \s{IS} on \s{T}. Then \Stoke~\cite{Stoke2013} is used  to synthesize  instruction sequences which contain instructions from \s{B} and match the behavior of \s{IS} for all test cases in \s{T}. Given two such generated instruction sequences \s{IS} and \s{IS}$^\prime$, their equivalence is decided  using an SMT solver and the trusted and known  semantics from the base set. If the two sequences are
semantically distinct, then the model produced by the SMT solver is used  to obtain
an input \s{t} that distinguishes \s{IS} and \s{IS}$^\prime$, and \s{t} is added to \s{T}. This process of synthesizing instruction sequence candidates and accepting or rejecting them based on equivalence checking with previous candidates, is repeated until a threshold is reached, which in their implementation is based on the number of accepted instruction sequences. 

%% Generalization 
\cmt{
Using the above technique, they first came up with the semantics of $692$ register
and $120$ immediate instructions. Then they used ``generalization'' of the register
instructions to get a total support count of \strataWithDupIS. Generalization is based on their hypothesis that the memory or immediate variants will behave identically with corresponding register variant (other than where the inputs come from) and hence can use the same formula as register variants. They validate this hypothesis using random testing. }

%% Two format of output
For each instruction, \Strata manifested its semantics in terms of two related artifacts.
The first artifact is an instruction sequence and the second is a set of SMT formulas 
in the bit-vector theory, one for each output register. 
The second is obtained by symbolically executing the first.

\cmt{
\Stoke~\cite{Stoke2013} contains manually written specifications  for a subset of the
x86-64 instruction set and there are APIs to generate SMT formulas from those specifications. 
\Strata uses those formulas to compare against the ones learned
by stratification by asking an SMT solver if the formulas are equivalent. For
example, for the instruction \instr{andnq \%rdx, \%rcx, \%rbx} the manually
written specification looks like the following:

\begin{lstlisting}[style=C++, label={lst:CS5}, caption={Manually Writen  specification for \instr{andnq
\%rdx, \%rcx, \%rbx}}]
semantic_function (
  Operand dst, Operand src1, Operand src2, 
  // dst == %rbx, src1 == %rcx, src2 == %rbx
  SymBitVector a, SymBitVector b, SymBitVector c, 
  // a, b, c are symbolic values for dst, src1, 
  // src2 resp. 
            SymState& ss) {
  // The symbolic state to be updated with the 
  // behaviour of the instruction     
  auto tmp = (!b) & c;
  
  // Setting destinaton
  ss.set(dst, tmp); 
  // Setting of, cf to false 
  ss.set(eflags_of, SymBool::_false()); 
  ss.set(eflags_cf, SymBool::_false());
  // Updating sf, zf based on result
  ss.set(eflags_sf, tmp[dst.size()-1]);
  ss.set(eflags_zf, tmp == SymBitVector::constant(dst.size(), 0));
  // Setting af, pf to undefined values
  ss.set(eflags_af, SymBool::tmp_var());
  ss.set(eflags_pf, SymBool::tmp_var());
}
\end{lstlisting}
}

\subsection{McSema}\label{sec:McS}



\section{Single Instruction Validation}\label{sec:siv}

The \siv is responsible for validating the lifting (using McSema) of an \ISA 
instruction \s{I} to \LLVM sequence \s{S}. This is achieved by (1) 
%Identifying the input/output variables for \s{I} and \s{S} and 
Establishing variable correspondence between \s{I} and \s{S}, (2) Generating 
symbolic summaries 
individually for \s{I} and \s{S} for each output variable, (3) Generating 
verification conditions meant to establish semantic equivalence between the 
corresponding pair of summaries, and solving those using an SMT solver (\Z). 
Next, we describe each one of these steps.

\paragraph{(1) Establishing variable correspondence:}   
%First, we need to identify the input/output variables of an \ISA instruction 
%and the corresponding lifted IR sequence. For each \ISA instruction, this 
%information is ready accessible using Stoke libraries to determine what are 
%the 
%implicit and explicit 
%register/memory/flags which are read or written by the instruction.
``Variable correspondence'' between \s{I} and \s{S} refers to identifying the 
correspondence between the input/output variables of \s{I}\footnote{By 
input/output variables of an instruction we mean implicit and explicit 
    register/memory/flags which are read or written.} and the 
    virtual registers in 
\s{S}. As described in Section~\ref{par:mcsema}, \mcsema uses a \Mcstate 
structure to
model the architecture state, which holds all the simulated architectural
registers at different offsets of the structure.\footnote{Another decompiler,
  fcd~\cite{FCD}, also has the similar approach of modeling.
    Rev.Ng~\cite{DiFederico:CC2017} models the architecture registers as LLVM
    globals.}  Hence, the input and output variables in the context of
    \mcsema are particular \emph{struct} fields, identified by constant offsets.
%
 As an example, for an instruction \instr{adcq \%rax, \%rbx}, the input variables are
 \reg{cf}, \reg{rax} \& \reg{rbx}, and output variables are \reg{rbx}, \reg{cf},
 \reg{pf}, \reg{sf}, \reg{zf}, \reg{of} and \reg{af}. 
The following shows how
 these input/output registers are mapped to the \mcsema \Mcstate structure.

\begin{lstlisting}[style=KRULE]
// State structure type (irrelevant fields shown as ...)
%struct.State (*$\mapsto$*) type { %struct.ArchState, ..., 
    %struct.ArithFlags,..., ..., ..., %struct.GPR, ...}

// Pointers to simulated registers are accessed as below
getelementptr inbounds %struct.State, %struct.State* 
    %state, i64 0, i32 (*\textbf{m}*), i32 (*\textbf{n}*), i32 0, i32 0

// Mapping of various simulated registers to getelementptr offsets
    rax (*$\mapsto$*) m = 6  n = 1;  rbx (*$\mapsto$*) m = 6  n = 3
     cf (*$\mapsto$*) m = 1  n = 1;   pf (*$\mapsto$*) m = 1  n = 3
     af (*$\mapsto$*) m = 1  n = 5;   zf (*$\mapsto$*) m = 1  n = 7
     sf (*$\mapsto$*) m = 1  n = 9;   of (*$\mapsto$*) m = 1  n = 13
\end{lstlisting}

We use the above architectural state representation of \mcsema to infer the
``variable correspondence'' between the \ISA instruction and its corresponding
lifted IR sequence\footnote{Similar inference
    for fcd~\cite{FCD}. For Rev.Ng~\cite{DiFederico:CC2017}, ``variable
    correspondence''  refers to  mapping between the \ISA registers and the
    LLVM globals}.

\cmt{  
For \mcsema, the ``variable correspondence''  means identifying the one-on-one
mapping between the \ISA registers and the \emph{struct} offsets.
%
\footnote{Same
  with fcd~\cite{FCD}. For Rev.Ng~\cite{DiFederico:CC2017}, ``variable
    correspondence''  refers to  mapping between the \ISA registers and the
    LLVM globals},}
%
% which is a trivial task once the input/output variable are already identified.
%
%\todo[inline,color=yellow]{I think we should use "variable correspondence" to 
%mean the mapping
%of operands of I to operands of S for each I in the input program. The mapping 
%above 
%is an architectural state representation, not a variable correspondence.  Your 
%%%tools 
%use the knowledge of the
%latter to extract the former automatically for each I.  Need to add a brief 
%paragraph here to describe that.}

%One may very rightfully think that it may not be easy or scalable to look into 
%the 
%implementation details of an arbitrary decompiler to collect such information. 
%To avoid that we developed an automated tool which will collect the such 
%variable correspondence information. The idea is to feed carefully selected 
%\ISA instructions to the  
 
\paragraph{(2) Generating symbolic summaries:}
The \K framework takes the formal semantics of \ISA and \LLVM and generates 
symbolic execution engines automatically, which we leverage to do 
symbolic execution of an \ISA instruction and the corresponding lifted \LLVM sequence 
individually. Before symbolic execution, we assign symbolic values to the input 
variables to obtain a summary over those. For the running example of 
\instr{adcq, \%rax, \%rbx}, the following shows the symbolic summary for just the 
output register \reg{rbx}\footnote{All the values or
    addresses, stored in registers, memory or
    flags, are represented as bit-vectors which are depicted in
    this paper as $V_W$ and interpreted as a bit-vector of size $W$
    and value $V$.}. 

\vspace{45pt}
%\todo[inline,color=yellow]{Usual notation is with a footnote: $V_W$. You could 
%use footnotes
%in formatted text and use W.V in the listings.  W.V makes the listing
%below very cluttered, though: wish there was a better choice.}

%\begin{minipage}{\linewidth}
%\vspace{10pt}
\begin{lstlisting}[style=KRULE]
// VX_CF, VX_RAX and VX_RBX are the symbolic values
// assigned to input variables.
extract ( 
    add ( 
        (#if eq ( VX_CF(*$_1$*) , (*$1_1$*) ) #then 
            add ( concat ( 0(*$_1$*) , VX_RAX(*$_{64}$*) ) , 1(*$_{65}$*) ) 
        #else 
            concat ( 0(*$_1$*) , VX_RAX(*$_{64}$*) ) 
        #fi)
        , concat ( 0(*$_1$*) , VX_RBX(*$_{64}$*)) 
        ) 
    , 1 , 65 ) 
\end{lstlisting}
%\end{minipage}

Similar symbolic summaries will be obtained for
 every simulated register in the lifter IR sequence,
which is omitted for brevity.

%%
\cmt{and 
(3) The fact that x86-64 ISA is largely stable and changes slowly
over time, we can keep a database of \emph{most} of the (\ISA
instruction,
validated IR sequence) pairs computed offline, but one-time, and
the phase two, on the other hand, can use the database for
composition.  Some
instructions variants like immediate, memory and control-flow cannot
be stored before-hand because it is impractical to compute the IR
sequence for all possible constant values. In these case, the IR
sequence can either be generalized from similar instances already
populated in the database or generated afresh and validated on the fly.}

The \ISA ISA includes instructions with Repeat String Operation 
Prefix (e.g. \instr{rep}, \instr{repz} etc.) to repeat a string instruction the 
number of times specified in the 
count register or until the indicated condition by the prefix is no longer met.
That is, their specification involves a loop which the symbolic 
execution must handle. We address this by symbolically executing those 
instruction with symbolic input state and comparing the summaries (using 
solver checks) of any single $i^{th}$ iteration of the two loops. This suffices 
to establish equivalence between the two loops, by coinductive 
reasoning~\cite{bisimulations} and the fact that such loops are bounded by a 
constant thus must terminate.

%\todo{"Capturing core behaviors" is not enough for soundness: can we make
%    a stronger argument?}

% Although we can only see a limited number of 
%iterations, this is sound as a single symbolic iteration is enough to 
%correctly capture the core behaviors of such instructions.

% how is this diff from Meandiff
% Meandiff converts the ind. iR to UIR and then to z3 quesries.
% wheeras we use the correctby construction K symbolic ex for the purpose.

\paragraph{(3) Generating \& Solving the verification conditions:}

First, we convert the summaries written in \K builtin operators to SMTLIB 
expressions. Given two symbolic summaries sum$_{\ISA}^{rbx}$ and 
sum$_{ir}^{rbx}$ for output \ISA register \reg{rbx} and corresponding 
simulated register, we emit a  query 
%\begin{center}
%\begin{tabular}{c}
\begin{lstlisting}[style=KRULEWOBORDER]
            (assert (not (= sum(*$_{\ISA}^{rbx}$*) (*sum$_{ir}^{rbx}$*))))
\end{lstlisting}
%\end{tabular}
%\end{center}

Similar  queries are generated for all registers, memory and 
flags (for examples, refer to~\cite{Suppl}). Note that we
generate queries for all registers/flags, not just the
ones clobbered, because the registers and flags not modified by
the instruction should have equivalent summaries (which is the unmodified value
    of the input symbolic value).

The verification condition queries are then dispatched to the \Z solver. If any 
two summaries 
fail to match, we have found a bug in McSema.

%\paragraph{\TV of instructions accessing data sections}

%\paragraph{Dispatching queries to solver}
Note that, even though we are using solver checks during the first phase, this
should not hamper the scalability of our program validation pipeline for 
the following
reasons.
First, the instruction-level validation is done for each instruction. 
Thus its verification condition is much simpler than that of whole program-level 
validation.
Second, the validation result of each instruction can be reused 
within a program  or across different programs, thus the validation cost can be 
amortized, or, done offline. 

%(1) Instruction lifting is checked one at a time and hence the solver
%queries are simple, (2) Most \ISA instructions do not have loops
%in their semantics (and the few that do are converted to a loop-free form),
%and therefore checking their equivalence is much easier than
%checking program-level equivalence using heavy-weight equivalence checkers.

\section{Program-Level Validation}\label{sec:plv}
The goal of \plv is to validate the translation of the input \ISA program
\s{P} to the Mcsema-lifted \LLVM program \s{T}.  Towards that goal, the first step
is to construct an alternative program \s{T$^\prime$} generated using the \compd
(Section~\ref{sec:compd}), which are then compare for syntactic equivalence 
using (Section~\ref{sec:matcher}).
%\todo{the last part of this sentence is confusing.  do you mean syntactic 
%checking?}

\subsection{Compositional Lifter}\label{sec:compd}
The \compd is responsible for generating the proposed \LLVM \s{T$^\prime$} by
composing the validated McSema-lifted IR sequences of the constituent binary
instructions of the \ISA program \s{P}. Importantly, the \compd design 
(Algorithm~\ref{alg:compd}) is simple---and took us about 
three man-weeks to implement. 

\s{P} is disassembled (line 2) to
identify function boundaries, and to decode 
instructions.
%Identifying functions is
%important because the Matcher (Section~\ref{sec:matcher}) will work at
%function-level granularity\todo{SD:makes sense?  CF: unclear whether this was 
%%%fundamental or just a design choice}. 
\cmt{Moreover, the framework 
is
plug-and-play in
using disassemblers other than ObjDump (line 2), like Intel's  XED~\cite{xed}.}
If the disassembled instruction $I_{disass}$ is already in Store, then its
corresponding (validated) IR sequence is reused.
Otherwise $I_{disass}$
is assembled (line 6) and lifted (using \mcsema) to
generate an \LLVM sequence that will be validated using Phase 1. The validated IR 
sequences are then composed (line 17) following program order.

\begin{algorithm}
    \SetKwInOut{Input}{Inputs}
    \SetKwInOut{Output}{Output}
    %\TitleOfAlgo{How to write algorithms}
    %\underline{function Euclid} $(a,b)$\;
    \Input{ \\
     \textbf{P:} \ISA binary program. \\
     \textbf{Store:} Validated pairs (<\s{I}, \s{S}> ) of instruction \s{I}
        and
        lifted IR sequence
        \s{S}. (possibly empty) \\
     \textbf{R:} Address Relocation information of binary P.
    }
    \Output{Lifted IR Program \s{T$^\prime$}}
    \BlankLine
    $T^\prime \gets \phi$ \\
    $P_{disass} \gets$ ObjDump($P$) \\
    \ForEach{\textup{function} $F_{disass}$ \textup{in} $P_{disass}$}{%
    \ForEach{\textup{instruction} $I_{disass}$ \textup{in} $F_{disass}$}{%
        \uIf{$I_{disass}$ \textup{not in} Store}{%
            $I \gets$ Assembler($I_{disass}$) \\
            $S \gets$ McSema($I$) \\
            Perform \TV of $I$ and $S$ (Phase 1) \\
            \uIf{\textup{Validation successful}}{%
            Add $<I_{disass},S>$ to $Store$
            }
            \Else {
            Report Bug
            }

        }
        \Else {
            \textup{Extract} $S$ \textup{from} $Store$ \textup{for}
            $I_{disass}$ \\
        }
        $T^\prime \gets$ \textup{Compose($T^\prime, S, R$)}
    }
    }
    \KwRet{$T^\prime$}
    %\NoCaptionOfAlgo
    \caption{\textbf{Compositional Lifting}}\label{alg:compd}
\end{algorithm}

\paragraph{Single instruction validation of control-flow instructions (line
8)}\label{sec:sivcntrl}

The \siv strategy described in Section~\ref{sec:siv} cannot be applied naively  
to control flow instructions. This is because the instruction fed to \mcsema 
(line 7), for \siv, is obtained from the disassembled instruction \s{I$_{disass}$},
wherein the relative offsets of binary jump/call instructions are specified as labels. 
Hence, the binary instruction \s{I} which we obtain from assembling \s{I$_{disass}$} 
(line 6), without program context, has an incorrect relative offset, which gets 
propagated to the lifted IR \s{S}.

We get around this problem by symbolically executing \s{I} with 
symbolic values 
assigned to the current PC and label.
That way, we get symbolic
summaries agnostic of the actual (incorrect) value of the relative offset.
Similarly, we symbolically execute  lifted IR sequence by assigning symbolic
values to the virtual register holding the simulated relative offset.

%Translation validation of control-flow instructions like \instr{jmp label} and
%\instr{call label} w/o the program context is a bit tricky because the behavior
%of these instruction are not context-free. For example, the behavior of
%\instr{jz rel\_off}, which updates the PC value to either PC +
%sizeInBytesOf(\instr{jz rel\_off}) or PC + sizeInBytesOf(\instr{jz rel\_off}) +
%rel\_off, is based on the value of current PC value \cmt{and the rel\_off}
%which depends on the position of the instruction in the \emph{.text} section of
%binary.

\paragraph{The ``Compose'' step}
Below we describe the step ``Compose'' (line 17), responsible for
composing the IR sequences together, using a few example binary
instructions.
%\todo{say, at a high level, what is going on with compose?  E.g., ``At a 
%high-level, Compose concatenates the instructions that were validated in Phase 
%1, based on the program P.''}

The composed program is initially empty. Upon encountering a function label, we
append the following code to it\footnote{\emph{mem} is pointer to an opaque 
struct type which together with return type allows ordering of memory
    operations if required.}.

\begin{lstlisting}[style=LLVM]
define %struct.Mem* @composedFunc(%struct.State*, i64, 
        %struct.Mem* mem)  {}
\end{lstlisting}

For an instruction \instr{adcq \%rax, \%rbx}, \mcsema generates the following
IR sequence.

\begin{lstlisting}[style=LLVM]
define internal %struct.Mem* @ADCImpl(
    %struct.Mem*, %struct.State*, i64*, i64, i64) {
    ; Does adc computation and updates destination RBX
    ; and flags (omitted for brevity)
}

define %struct.Mem* @sub_adcq_rax_rbx(%struct.State*, 
        i64, %struct.Mem* ) {
 %RIP = getelementptr ... ; Compute simulated RIP address
 %RAX = getelementptr ... ; Compute simulated RAX address
 %RBX = getelementptr ... ; Compute simulated RBX address
 %VAL_RBX = load i64, i64* %RBX
 %VAL_RAX = load i64, i64* %RAX
 ; RIP update based on instruction size
 %VAL_RIP = load i64, i64* %RIP
 %UPDATED_RIP = add i64 %VAL_RIP, 3
 store i64 %UPDATED_RIP, i64* %RIP
 %retval = call %struct.Mem* @ADCImpl(
        %struct.Mem* %2, %struct.State* %0, i64* %RBX,
        i64 %VAL_RBX, i64 %VAL_RAX)
 ret %struct.Mem* %retval
}
\end{lstlisting}

The above IR sequence is then appended to the composed program as below.

\begin{lstlisting}[style=LLVM]
define %struct.Mem* @composedFunc(%struct.State*, 
        i64, %struct.Mem* mem)  {
    ; Code: adcq %rax, %rbx	
    %loadMem = load %struct.Mem*, %struct.Mem** %mem
    %retval = call %struct.Mem* @routine_adcq_rax_rbx(
        %struct.State* %0, %struct.Mem* %loadMem)
    store %struct.Mem* %call, %struct.Mem** %mem

    ret %struct.Mem* retval
}
; Definitions of called functions omitted for brevity
\end{lstlisting}

A similar composition happens for most instructions, the exceptions being the control-flow
data section-accessing instructions, which we elaborate on next.

\paragraph{Composing control-flow instructions} As mentioned previously, the
``labeled'' control-flow assembly instructions, when assembled without program
context, generate incorrect offsets which get propagated to the lifted IR.
We fix this IR by replacing said incorrect relative offsets with the correct
offsets.

For instance, when \mcsema lifts \instr{jne .L\_40087e} in isolation,
it generates the following IR sequence:
%\vspace{10pt}
\begin{lstlisting}[style=LLVM]
define %struct.Mem* @sub_jne_.L_40087e(%struct.State*, 
        i64, %struct.Mem* ) {
  %RIP = getelementptr ... ; Compute simulated RIP address
  %RIP_VAL = load i64, i64* %RIP
  ; Compute true target (using incorrect offset)
  %TARGET1 = add i64 %RIP_VAL, <incorrect_val>
  ; Compute fall-through target
  %TARGET2 = add i64 %RIP_VAL, <instr. size>
  %retval = call %struct.Mem* @JNEImpl(..., i64 %TARGET1, i64 %TARGET2)
  ret %struct.Mem* %retval
}
\end{lstlisting}

And the composed program with transformed IR looks like
\begin{lstlisting}[style=LLVM]
define %struct.Mem* @sub_jne_.L_40087e(%struct.State*, 
        i64, %struct.Mem*,
        i64 %(*\textbf{true\_tgt}*), i64 %(*\textbf{false\_tgt}*)) {
  %RIP = getelementptr ... ; Compute RIP address
  %RIP_VAL = load i64, i64* %RIP
  ; Transformed code
  %TARGET1 = add i64 %RIP_VAL,  %(*\textbf{true\_tgt}*)
  %TARGET2 = add i64 %RIP_VAL,  %(*\textbf{false\_tgt}*)
  ; Rest same as above
}

define %struct.Mem* @composedFunc( %struct.State*, 
        i64, %struct.Mem* mem)  {
  ; ... previously composed code ...
  ; Code: jne .L_40087e	 RIP: 400855	 Bytes: 6
  %loadMem = load %struct.Mem*, %struct.Mem** %mem
  %retval = call %struct.Mem* @routine_jne_.L_40087e(
        %struct.State* %0, %struct.Mem* %loadMem, 
        i64 (*\textbf{41}*), i64 (*\textbf{6}*))
  store %struct.Mem* %retval, %struct.Mem** %mem
  ret %struct.Mem* %retval
}
\end{lstlisting}

\paragraph{Composing data-section access instructions}
Instructions accessing the data section, like \instr{movq 0x602040, \%rdi}
with the first operand being an address, cannot be lifted correctly in
isolation (without program context) because \mcsema does not have sufficient information to to distinguish constant from address. 
\Siv can only validate the fact whether the constant
(which could potentially be an address) is correctly moved to the destination register. 
However, the problem is the \plv cannot use that lifting because the resulting 
composed IR \s{T$^\prime$}, with a constant moved to \reg{rdi}, will be   
different from the
one lifted by \mcsema \s{T}, with a global address moved to \reg{rdi}.
Upon normalization, two such IRs will be optimized differently by LLVM, leading to two syntactically 
divergent normalized forms, even when the initial programs were equal.

%Note that this will not pose any
%problem while \siv, as we can validate just the fact whether the constant
%(which could potentially be an address) is moved to the destination register.
%However, the \plv cannot use that lifting because the resulting composed IR
%\s{T$^\prime$}, with a constant moved to \reg{rdi}, will be different from the
%one lifted by \mcsema \s{T}, with a \dlifted global address moved to \reg{rdi}.
%Two such IRs upon normalization (Section~\ref{sec:normalizer}) using LLVM
%optimizer
%allows different optimization opportunities leading two syntactically divergent
%normalized forms.

To aid in testing, we compile binaries with options to retain auxiliary
information. To disambiguate between cases where a constant is a reference
into the data section (e.g., an \texttt{int*}) v/s a scalar (e.g., an
\texttt{int}), we use relocation information, denoted by \s{R} in
algorithm~\ref{alg:compd}.  We allow \mcsema to (incorrectly) lift such
instructions in isolation and then we course-correct the lifted IR by
consulting the binary's relocation information to determine if an immediate
operand should be considered as an address or constant --- every immediate
operand that is a reference has a corresponding entry in the relocation table.
Missing this entry would automatically mean that the immediate is a constant.

For example, the incorrect IR generated by \mcsema when lifting \instr{movq
0x602040, \%rdi} in isolation is:
\begin{lstlisting}[style=LLVM]
define %struct.Mem* @sub_movq_0x602040___rdi(
        %struct.State*, i64, %struct.Mem* ) {
    ...
    %retval = call %struct.Mem* @MOVImpl(
        %struct.Mem* %2, %struct.State* %0,
        ; data-section addr 0x602040
        ; lifted as a constant
        %i64* %RDI, i64 6299712)

    ret %struct.Mem* %retval
}
\end{lstlisting}

The address relocation information in the binary allows us to identify the address
and the following correct lifting:
\begin{lstlisting}[style=LLVM]
%G_0x602040_type = type <{ [8 x i8] }>
@G_0x602040= global %G_0x602040_type zeroinitializer
define %struct.Mem* @sub_movq_0x602040___rdi(
        %struct.State*, i64, %struct.Mem* ) {
    ...
    %retval = call %struct.Mem* @MOVImpl(
     %struct.Mem* %2, %struct.State* %0,
     %i64* %RDI,
     i64 ptrtoint( %G_0x602040_type* @G_0x602040 to i64))

    ret %struct.Mem* %retval
}
\end{lstlisting}

We reiterate that \compd only uses relocation information to strengthen the
generated golden reference, \s{T$^\prime$}, when such information is
available, e.g., during test or development time. This allows for a tighter
specification, allowing our technique to find bugs at testing that would 
otherwise
be missed. During use of \compd in the field to validate the lifting of
\mcsema on an unknown, blackbox binary, we do not require this additional
information, at the cost of potentially missing bugs described above. Note
that this is a fundamental limitation because \ISA semantics for an
instruction has no notion of types, and therefore \s{T$^\prime$}, which is
based on \ISA semantics, should allow for the ambiguity and cannot enforce
stricter type requirements. McSema on the other hand is never given this
additional information as it is expected to work in the field where relocation
information is rarely available, except in library code.

%\subsection{Normalizer}\label{sec:normalizer}

Algorithm~\ref{alg:NM} summaries the normalization and subsequent matching 
phase.
Due to the nature of the composition, the composed program  is 
very similar to the lifted program. We leverage this observation to establish 
semantic equivalence between the two 
programs using an iterative matching and pruning strategy, realized by a tool 
we develop called the \matcher. Failure of the  
\matcher should be interpreted as a \emph{potential} bug in the lifted program. 
%comparing (using \emph{Matcher}) 
%the \emph{canonical 
%representations} of the input programs
%generated using LLVM  \emph{O3} passes \& iterative pruning. 
%uisng an iterative strategy of matching the LLVM 03 assisted canonicalized 
%program and sbsequent pruning.
\newcommand\mycommfont[1]{\footnotesize\textcolor{blue}{#1}}
\SetCommentSty{mycommfont}
\begin{algorithm}
    \SetKwInOut{Input}{Inputs}
    \SetKwInOut{Output}{Output}
    \Input{
        \textbf{T:} \dlifted IR. \\
        \textbf{T$^\prime$:} \compd lifted IR.
    }
   \Output{\textbf{True} $\implies$   \textbf{T} \& \textbf{T$^\prime$} 
   semantically equivalent \\
   \textbf{False} $\implies$   \textbf{T} \& \textbf{T$^\prime$} \emph{may-be}
   non-equivalent
}


    \BlankLine
    
    \ForEach{\textup{corresponding function pair (\F,\FP) in 
        (\T, \TP)}}{%
        
        \If{\textup{!Matcher(\F, \FP)}}{%
            \tcp{\textup{A potential bug in McSema while lifting \s{F}}}
            \KwRet{false}  \\ 
        }
        
    }
    \KwRet{true}
    %\NoCaptionOfAlgo
    \caption{\textbf{Normalization \& Matching}}\label{alg:NM}
\end{algorithm}
%\Output{ 
%    \textf{True} $\implies$  \textbf{T} \& \textbf{T$^\prime$} semantically 
%    equivalent \\
%    False: \textbf{T} \& \textbf{T$^\prime$}  equivalent
%}



%two programs we are comparing (\s{T} \& \s{T$^\prime$}) start out being very
%similar:   
%% Why we need a normalizer
%There is no control-flow changes 
%
%\mcsema generally does not change control flow, and does not add or remove
%almost \emph{any} operations. However, it hoists the address computations of
%all the simulated registers (using \emph{getelelementptr}) in the \emph{entry}
%basic block so that the simulated instruction semantics does not have to
%compute them again. Whereas, the \compd recomputes those address at every
%instruction site.  
%%
%Our approach is to leverage this observation and to syntactically compare
%\emph{canonical representations} of the input programs.  We generate such
%representations using a tool we develop called a \emph{normalizer}.  Our
%normalizer is current implemented using LLVM  \emph{O3} passes to be applied to
%both \s{T} \& \s{T$^\prime$}.




%During the canonicalization step performed by the compiler, the strand (in our
%case a procedure with a single basic block) is repre- sented by a directed
%acyclic graph (DAG) which stores the expression. Even though comparing DAGs is
%possible, we wanted to simplify our representation to some kind of tex- tual
%form, allowing for fast and simple comparison. This is accomplished by using
%opt to output a linearized version of the computation’s DAG. To finalize the
%transformation which eliminates the origin and compilation choices made in the
%creation of the binary code, the final refinement to our representation is
%normalizing the strands. This is done by re- naming all symbols in the strand,
%i.e., its registers and tem- porary values, into sequentially named symbols.
%This step is crucial for cross-architecture comparison, as the names of the
%specific registers used in a given computation have nothing to do with its
%actual semantics, and are completely different between architectures.

\subsection{Matcher}\label{sec:matcher}
% How does the normalized outs look like
% Choice os SSA graph -> Why not conflow graph
% Why the matcher + Semantics Preserving Transformation is sufficient.
Algorithm~\ref{alg:NM} summarizes our overall strategy to check equivalence 
between the IRs generated by \mcsema (\T) and \compd (\TP). Due to the nature 
of the composition, \T \& \TP are structurally very similar. 
We leverage this observation to establish 
semantic equivalence between the two using a graph-isomorphism based 
algorithm assisted by 
normalization.  The algorithm is realized by a tool 
we develop called the \matcher. If the   
\matcher fails to match \T \& \TP, there is a \emph{potential} bug in the lifted program. 
%comparing (using \emph{Matcher}) 
%the \emph{canonical 
%representations} of the input programs
%generated using LLVM  \emph{O3} passes \& iterative pruning. 
%uisng an iterative strategy of matching the LLVM 03 assisted canonicalized 
%program and sbsequent pruning.
\newcommand\mycommfont[1]{\footnotesize\textcolor{blue}{#1}}
\SetCommentSty{mycommfont}
\begin{algorithm}
    \SetKwInOut{Input}{Inputs}
    \SetKwInOut{Output}{Output}
    \Input{
        \textbf{T:} \dlifted IR. \\
        \textbf{T$^\prime$:} \compd lifted IR.
    }
    \Output{\textbf{True} $\implies$   \textbf{T} \& \textbf{T$^\prime$} 
        semantically equivalent \\
        \textbf{False} $\implies$   \textbf{T} \& \textbf{T$^\prime$} 
        \emph{may-be}
        non-equivalent
    }
    
    
    \BlankLine
    
    \ForEach{\textup{corresponding function pair (\F,\FP) in 
            (\T, \TP)}}{%
        
        \If{\textup{!Matcher(\F, \FP)}}{%
            \tcp{\textup{A potential bug in McSema while lifting \s{F}}}
            \KwRet{false}  \\ 
        }
        
    }
    \KwRet{true}
    %\NoCaptionOfAlgo
    \caption{\textbf{Matcher Strategy}}\label{alg:NM}
\end{algorithm}

The matcher algorithm is based on the following key observations on input IR 
programs, \s{T} \& \Tp, informally stated:
%
(I) Both exhibit identical control-flow and identical sequences of memory allocation
and reference behaviors (because McSema does not modify control flow or memory operations
during lifting).
%
(II) The single-instruction validation step proves that a memory store 
(respectively, load) in \s{T} writes (resp., reads) the equivalent set of memory
locations as does the corresponding operation in \Tp.  This property holds for
each dynamic instance of the corresponding instructions.

%
%, and 
%(III) There is no alloca instruction in either \T or \TP. All the load 
%(store) 
%instructions are reading from 
%(writing to) the Mcsema \Mcstate fields. This makes sense because for each 
%input \ISA instruction, both \T \& \TP simulates its read or write behavior on 
%register/flags/memory which are all modeled as fields in the \Mcstate 
%structure.

%Two instructions (one from each input program) which should match, as per its 
%data-flow behavior, may not occur in  their respective matching basic blocks. 
%This 
%is because  As the two input program 
%are not exactly equal to begin with, the instruction can get reordered 
%differently,

These two observations motivate an intuitively simple graph isomorphism strategy for
proving the equivalence of \T and \Tp.
%
Let us name the normalized versions of the function pair, \F \& \FP, as \FN \& 
\FNP.
The Matcher algorithm works on data dependence graphs, \GN \& \GNP, generated  
from \FN \& \FNP. A vertex of the graph represents an 
LLVM instruction and an edge between two vertices captures SSA def-use  or   
memory 
dependence relations. Memory dependence edges, extracted from alias analysis 
results, appear between LLVM load and 
store instructions.
\cmt{There is a particular reason why we do not add 
control-flow edges: It is evident from \emph{\textbf{Ob II}} that the input 
programs \T \& \TP, being not exactly equal to begin with, can be normalized 
differently and there is no guarantee that two matching instructions will end 
up in matching basic blocks. }
%\cmt{We 
%call an 
%instruction in \s{N} \emph{matching exactly} with 
%an instruction in \s{N$^\prime$}  if the containing sub-graphs are 
%isomorphic.} 

\paragraph{Soundness of Equivalence via Graph Isomorphism}
%
We argue informally that isomorphism of \GN \& \GNP implies semantic equivalence of
the programs \s{T} and \Tp. We consider all of memory used in an execution as a single
``SSA variable,'' which gets renamed at every store operation in the program.  A store
modifies some (unknown) subset of the locations in memory, and a load reads some
(unknown) subset of the bytes.  Given two isomorphic graphs \GN and \GNP, consider a 
matching pair of nodes representing a store instruction \s{S} in \s{T} and the 
corresponding store \Sp in \Tp.  A key to the correctness argument, below, is that
single-instruction validation proves that, if the initial state of memory and registers
is identical before executing \s{S} and \Sp, then the final state of memory and registers
is also identical, i.e., the same bytes have been written into the corresponding memory
locations.  Similarly, a matching pair of load instructions transfers identical bytes
from memory to SSA registers in \s{T} and \Tp.
%
The correctness argument then works as follows:
%
\begin{enumerate}
  %
  \item A node \s{N} and corresponding node \Np are equivalent because of the 
  equivalence proof constructed by single-instruction validation (Section~\ref{sec:siv}).
  %
  \item An SSA edge \s{A} $\rightarrow$ \s{B} and corresponding edge 
  \Ap $\rightarrow$ \Bp carry identical bytes of data, because \s{A} is equivalent
  to \Ap and \s{B} is equivalent to \Bp, by (1), above.
  %
  \item A memory edge \s{S} $\rightarrow$ \s{L} representing a \emph{true} memory
  dependence (i.e., a store-to-load dependence) and corresponding edge \Sp $\rightarrow$
  \Lp carry identical bytes of data, because \s{S} and \Sp store identical bytes into
  identical memory locations, and \s{L} and \Lp read identical bytes from identical
  memory locations.  The argument for \emph{anti} and \emph{output} memory dependences
  is similar.
  %
\end{enumerate}

Note that the above argument is independent of the precision of any static analysis
used to identify memory dependences.  A highly imprecise analysis (e.g., one that
says every store-load or store-store pair may be aliased) might lead to a failure to
prove isomorphism between \s{T} and \Tp, but will not claim isomorphism if the two
programs are not equivalent.  In practice, we find in our experiments, described in 
Section~\ref{sec:eval}, that the memory dependence edges from such a highly imprecise
analysis do indeed reduce the success rate of the Matcher, but only by a small amount.
A more precise analysis may improve the success rate, reducing the number of false
negatives.

\paragraph{Checking Graph Isomorphism}
%
We build on a subgraph-isomorphism algorithm from Saltz 
et al.~\cite{Saltz2014}, named 
\emph{dual-simulation},  to check if both \GN \& \GNP are 
subgraph-isomorphic to each other. The algorithm, in general, first 
retrieves initial potential match sets, $\Phi$,  for each vertex in one 
graph based on semantic and/or neighborhood information in the other graph.
In our case, the initial potential match set for an instruction 
\IN in \GN contains all the instructions in \GNP which have the same 
instruction opcode. Also, if \IN has constant operands then its potential 
matches must share 
those.  
Then the algorithm iteratively prunes out elements from the 
potential match 
set of each vertex based on its parents/child relations until it reaches a 
fixed-point.
%
Therefore, nodes \s{A} and \Ap in \GN and \GNP will be marked as isomorphic if
they have identical (isomorphic) sets of predecessors and successors.
%
Two edges will be marked as isomorphic if their source and sink nodes are isomorphic.


%\cmt{Then we augmented the algorithm to infer the basic-block 
%correspondence 
%on the fly and use that information to prune out the potential sets of store 
%instructions.}

%\paragraph{\textup{\GN} \& \textup{\GNP} are non-isomorphic} 
%This could happen because the input functions (\F \& \FP), being not exactly 
%equal to begin with (from Obs. II), undergo different 
%optimizations. 
%
%There is one significant difference in the code in \s{T} and \Tp.
%%
%In \T, the addresses computations, using 
%\emph{getelelementptr} (gep in short) instructions, of all the simulated 
%registers 
%and flags  
%are hoisted in the \emph{entry} basic block. \mcsema does this as an 
%optimization so that  the subsequent data-dependent  
%instructions does not have to compute them again. 
%Whereas, in \TP  addresses of relevant registers and/or flags are recomputed at 
%every instruction site.  The normalizer is quite effective at transforming the two
%code versions to be similar, so that the data dependence graphs are isomorphic.
%
%
%As a simple example, consider the code snippets of the normalized function pair 
%(\FN \& \FNP). \FN is generated by normalizing \F, wherein the  
%simulated register \& sub-register address computations  are hoisted in 
%\reg{entry} block and 
%re-used at use-site. Whereas, \FNP is generated from \FP where the 
%simulated address is re-computed 
%at every use-site. \F and \FP upon normalization undergo 
%different optimizations; One with better CSE (common subexpr. elim.) than 
%the other.   
%\begin{lstlisting}[style=LLVMWOBORDER]
%          (*\FN*)                            (*\FNP*)
%%entry:                           %entry:
%  %expr = gep ...                 %expr = gep 
%  %cl =  gep %expr ...            %cl =  gep %expr 
%  %rcx = gep %expr ...            %rcx =  gep ...   
%  ...                               ... 
%%somebb:                          %somebb:
%  store to %rcx                      store to %rcx
%  ...                                ...
%%otherbb:                         %otherbb:
%  store to %cl                       store to %cl  
%  ...                                ... 
%\end{lstlisting}
%Clearly, the  \FN \& \FNP are equivalent, yet the naive isomorphism based 
%matcher has to declare them as non-equivalent because the corresponding graphs 
%are
%not isomorphic with the node \reg{cmn\_expr} in \GN has two out-edges versus 1 
%edge in \GNP. However, the key insight is: the  subgraph with nodes 
%\reg{expr}, 
%\reg{cl} and \s{store \%cl} in \GNP shares no data-dependent edges with the 
%rest of the graph and is isomorphic with the corresponding subgraph in \GN. We 
%can prune both the subgraph from their respective parent graphs and the 
%residual graphs, upon re-normalization, have better opportunity to get 
%normalized to isomorphic graphs.
%
%Again, consider the code snippets below. 
%%As before, \F has the   
%%simulated address computation, hoisted  in \reg{entry}, re-used in all its 
%%use-site.  
%%Whereas, \FNP is generated from \FP where the simulated address is re-computed 
%%at every use-site. 
%\FP upon normalization undergo 
%partial-redundancy-elimination to make the computation of \s{rax} 
%\emph{available} in both the paths (\reg{b0} and \reg{b1}), but missed 
% the opportunity to eliminate the common-subexpression, despite of the fact 
% that \s{pre\_rax} has no data-dependence on ``some-code''.      
%
%\begin{lstlisting}[style=LLVMWOBORDER]
%         (*\FN*)                            (*\FNP*)
%%entry:                         %entry:
% %rax =  gep ...                 ...   
% ...
% br %some_cond, %b0, %b1         br %some_cond, %b0, %b1
%%b0:                           %b0:
%    .. some-code ..               .. some-code ..
%                                 pre_rax = gep ...
%    br %merge                    br %merge
%%b1:                           %b1: 
%    store to %rax                rax1 = gep ... 
%                                 store to %rax1   
%    ; ...                        ; ...
%    br %merge                    br %merge
%%merge:                        %merge: 
%    store to %rax                rax2 = (*$\phi$*) [rax1, %b1], 
%                                         [%pre_rax, %b0 ]
%                                 store to %rax2
%\end{lstlisting}
%As before, despite the equivalence of \FN \& \FNP, the naive 
%matcher will fail to proof graph-isomorphism, resulting in false alarm.
%However, if we find the sub-graphs corresponding to ``some-code'' on either 
%side are matching exactly, then we can follow  the pruning strategy as before, 
%followed by normalization, and converge to isomorphic graphs.
%
%With that insight, we design the following iterative matching and pruning 
%algorithm (Algorithm~\ref{alg:Match}).  
%\begin{algorithm}
%    \SetKwInOut{Input}{Inputs}
%    \SetKwInOut{Output}{Output}
%    \Input{ \\
%        \textbf{\F:} \dlifted function. \\
%        \textbf{\FP:} \compd lifted function.
%    }
%    \Output{Check if \textbf{\F} \& \textbf{\FP} are semantically 
%        equivalent}
%    \BlankLine
%        itr $\gets$ MaxIter \\
%        \While{\textup{itr != 0}} {
%            $\FN \gets \textup{llvm-opt -O3 } (\F)$ \\
%            $\FNP \gets \textup{llvm-opt -O3 } (\FP)$ \\
%            
%            \GN $\gets$  data-dependence graph of \FN \\
%            \GNP $\gets$ data-dependence graph of \FNP \\
%            
%            \If{\GN \& \GNP are isoporphic} {
%                \KwRet{true}
%            }
%            
%            $(\F, \FP) \gets \textup{Prune isoporphic subgraphs from \GN \& 
%            \GNP}$
%            \BlankLine
%            itr $\gets$ itr - 1 \\
%        }
%        \KwRet{false}         
%    %\NoCaptionOfAlgo
%    \caption{\textbf{Matcher}}\label{alg:Match}
%\end{algorithm}
%The Matcher algorithm, being agnostic of the normalization pass, tries to 
%recover missed optimization opportunities during normalization. This ensures 
%that there will be very less false alarms.

%\todo[inline]{The soundness problem that Vikram pointed out: Removing code
%    might introduce undefined behavior, which the later opt passes might abuse
%    and we might end up getting false positive or false negatives.}
%\begin{algorithm}
%    \SetKwInOut{Input}{Inputs}
%    \SetKwInOut{Output}{Output}
%    \Input{ \\
%        \textbf{\GN:} SSA graph of \N. \\
%        \textbf{\GNP:} SSA graph of  \NP.
%    }
%    \Output{Check if \textbf{T} \& \textbf{T$^\prime$} are semantically 
%        equivalent}
%    \BlankLine
%
%    changed $\gets$ true \\
%    \While{changed} { 
%       changed $\gets$ false \\
%       \For{\un $\gets$ \GN } {
%           \For{ \up $\gets$ \GN.adj(\un)} {
%               \potpup $\gets$ $\emptyset$ \\
%               \For{ \vn $\gets$ \potu } {
%                   \potvup $\gets$ \GN.adj(\vn) $\cap$ \potup \\
%                   \If{\potvup = $\empty$} {
%                       remove \vn from \potu \\
%                       \If{\potu = $\emptyset$} {
%                           \KwRet{$\emptyset$} \\
%                       }
%                       changed $\gets$ true \\
%                   }
%                   \potpup $\gets$ \potpup $\cap$ \potvup \\
%               }
%               \If{ \potpup = $\emptyset$} {
%                   \KwRet{$\emptyset$} \\
%               }
%               \If{ \potpup is smaller than \potup} {
%                   changed $\gets$ true \\
%               }
%               \potup $\gets$ \potup $\cap$ \potpup \\
%           }
%       }
%   }
%   \KwRet{$\emptyset$} \\
%    %\NoCaptionOfAlgo
%    \caption{\textbf{Dual Simulation}}\label{alg:DS}
%\end{algorithm}

 

\paragraph{Comparison with LLVM-MD \& Peggy}
At this point, it is important to differentiate our approach to establish 
equivalence between two \LLVM programs, using  normalization followed by 
matching, 
from some of the existing
approaches for validating LLVM IR-to-IR optimization
passes, e.g. LLVM-MD~\cite{Tristan:2011} and Peggy~\cite{Stepp:2011}, which, 
like our approach, move away from simulation proofs, and instead use graph 
isomorphism techniques to prove equivalence. 
Both build graphs of expressions for each program, 
transform the graphs via a series of ``expert-provided'' rewrite rules, and 
check for equality. The rewrite-rules mimic various compiler-IR optimizations 
and hence the technique is precise when the output program is an 
optimization of the input program and the optimizations are captured by the 
rewrite  rules. 

Compared to these approaches, our normalizer is simpler, requires no additional 
implementation effort, and re-uses off-the-shelf, well-tested compiler passes 
to reduce the two programs to syntactic equivalence. Nevertheless, the 
normalizer is still very effective as shown in our evaluations.

%In our case, \s{T} (the \dlifted program) and \s{T$^\prime$} (the \compd 
%lifted 
%program) are structurally separated by idioms which 
%might be beyond the capability of compiler optimization passes to match 
%%%syntactically.
%Encoding such idioms as rewrite rules would make the 
%approach tied to a specific lifter, which is something we avoided by 
%making the matcher iterative. \todo[inline]{too heavy-weight for our purpose}

%\paragraph{Semantics Preserving Transformation}




%Matching expressions with complex φ-nodes seems well within
%the reach of any SMTprover. Our preliminary experiments with Z3 suggest that
%it can easily handle the sort of equivalences we need to show. However, this
%seems like a very heavy-weight tool. One question in our minds is whether or
%not there is an effective tech- nique somewhere in the middle: more
%sophisticated than syntactic matching, but short of a full SMT prover.
%\begin{lstlisting}[style=LLVMWOBORDER]
%          (*\F*)                            (*\FN*)
%%entry:                                 %entry:
% ; chain of geps to step                  %comon_expr = gep ...    
% ; thr. nested State struct               %cl =  gep %common_expr  
% ; to compute address of cl               %rcx = gep %common_expr
% %cl =  gep ...                           ...
%
% ; chain of geps to step
% ; thr. nested State struct
% ; to compute address of rcx 
% %rcx =  gep ...  
% ...
%%somebb:                                %somebb:
% store to %rcx                            store to %rcx                        
% ...                                      ...  
%%otherbb:                               %otherbb:   
% store to %cl                             store to %cl
% ...                                      ... 
%\end{lstlisting}
%Next,  looks at the code snippets of \FP and its normalized version \FNP.
%\begin{lstlisting}[style=LLVMWOBORDER]
%          (*\FP*)                            (*\FNP*)
%%entry:                                 %entry:
%                                          %common_expr = gep ...
%                                          %cl =  gep ... 
%                                          %rcx =  gep ... 
% ...                                      ...
%%somebb:                                %somebb:
% ; chain of geps            
% %cl =  gep ...
% store to %rcx                            store to %rcx                        
% ...                                      ...  
%%otherbb:                               %otherbb:   
% ; chain of geps
% %rcx =  gep ...  
% store to %cl                             store to %cl
% ...                                      ... 
%\end{lstlisting}


\section{Evaluation} \label{sec:eval}
In this section, we present the experimental evaluation of 
\siv and \plv. All the experiments are run on an Intel Xeon CPU E5-2640 v6 at 3.00GHz
and an AMD  EPYC 7571 at 2.7GHz.  We aim to address three questions through these
experiments:
%
\begin{itemize}
  %
  \item[Q1.] Is single-instruction validation by itself useful for finding bugs in
  a sophisticated decompiler, even though no context information is used during 
  decompilation?
  %
  \item[Q2.] What fraction of function translations are successfully proven correct
  by \plv, and what is the false alarm rate of the tool?
  %
  \item[Q3.] Is \plv effective at finding additional potential bugs in a complex 
  lifter like McSema, beyond those found by \siv alone?  We studied this question
  using artificially injected bugs, because all real bugs were caught by \siv.
\end{itemize}

\paragraph{Usefulness of \siv}
The goal here is to validate the lifting of individual \ISA instruction to 
\LLVM sequences using \mcsema. 
Haswell \ISA ISA supports a total of \totalIS instruction variants of which 
\currentIS are formally specified in~\cite{DasguptaAdve:PLDI19}. \mcsema 
supports \mcsemaIS instructions all supported by ~\cite{DasguptaAdve:PLDI19}.
We had to exclude \sivExc instruction variants because of limitations 
of the \LLVM semantics~\cite{LLVMSEMA}, which does not support vector and floating 
points types and associated operations, and various intrinsics 
functions. (However, we do support intrinsics like llvm.ctpop, which is 
pervasively generated in the lifted IR for updating the \reg{pf} flag.) This 
brings us to a total of \sivIS viable instruction variants, and we apply
\TV to each of them individually.

Out of the \sivIS translation validations, \sivFail cases fail (hence are bugs) and 
\sivTO terminate with timeouts. The timeouts all correspond to \instr{paddb}, 
\instr{psubb}, and \instr{mulq} family of instructions, and are because of
complex summaries coming out of the lifted IR. For all the cases that timed out,
we manually inspected them to check that the generated code fragments are 
semantically equivalent.

The \sivFail failures were all reported 
as possible bugs~\cite{Suppl} to the \mcsema project, and all \sivFail 
have been confirmed as bugs by the \mcsema developers.
%
The following gives some brief examples of some of the
discrepancies we found.

First, \instr{xaddq \%rax, \%rbx} expects the operations 
(1) temp $\leftarrow$ \reg{rax} + \reg{rbx}, (2) \reg{rax} 
$\leftarrow$ \reg{rbx}, and (3) \reg{rbx} $\leftarrow$ temp, in that 
order. \mcsema performs the same operation differently as (A) old\_rbx = 
\reg{rbx}, (B) temp $\leftarrow$ \reg{rax} + \reg{rbx}, (C) \reg{rbx} 
$\leftarrow$ temp, and (D) \reg{rax} 
$\leftarrow$ old\_rbx. This will fail to  work when the operands 
are the same registers. 

Second, for instruction \instr{andnps \%xmm2, \%xmm1}, the Intel 
Manual~\cite{IntelManual} says the implementation should be \reg{xmm1}  
$\leftarrow$ $\sim$\reg{xmm1} \& \reg{xmm2}, whereas \mcsema 
interchanges the source operands. 

Third, for \instr{pmuludq \%xmm2, \%xmm1} instruction, both the higher and 
lower double-words of the source operands need to multiply, whereas 
\mcsema multiplies just the lower double-words.

Fourth, for \instr{cmpxchgl \%ecx, \%ebx}, \mcsema compares the entire $64$-bit 
\reg{rbx} (instead of just \reg{ebx}) with the accumulator 
\s{Concat(0x00000000, \reg{eax}}).

Finally, for \instr{cmpxchgb \%ah, \%al}, the lower $8$-bits of \reg{rax} 
should be replaced with the higher $8$-bits at the end of the instruction, whereas 
\mcsema keeps them unchanged.

%\todo{Is it mcsema to remill bugs}

%3062 - 159 - 2189
%714
%3062-714
%2348
%
\paragraph{Program-level validation: Success rate and false alarms:}
%
The goal here is to validate the translation of programs, one function at a 
time, using the Matcher strategy (Section~\ref{sec:matcher}). For this purpose, 
we use programs from LLVM-8.0 ``single-source-benchmarks''. The benchmark suite
consists of a total of $102$ programs, of which $11$ cannot be lifted by 
\mcsema due to missing instruction semantics. The remaining programs 
contain $3062$ functions in total. We excluded $714$ functions because  the 
corresponding binary uses floating point instructions which are not supported
in the LLVM formal semantics and so could not be validated using single 
instruction validation. This 
brings us to a grand total of \plvT usable functions which we compiles (using 
both gcc/clang) and feed the binaries to \compd and \mcsema for lifting. The 
length of (inlined) lifted IR functions ranges from $86-21729$, with an average 
of $777$.

Of the \plvT usable functions, our matcher can correctly and formally 
prove the correctness of the translations for \plvP using graph-isomorphism, 
i.e., a success rate of 93\%.
We manually checked the remaining $159$ and found them to be false alarms.
Overall, a false alarm rate of about 7\% is low enough that we believe
our Matcher can be of practical use for validation and testing of a decompiler.

\paragraph{Program-level validation: Effectiveness at finding bugs:}
%
In order to evaluate whether our strategy is effective in finding bugs in lifters like 
McSema, we manually injected bugs in their implementation. The injected bugs 
covers the following aspects of \mcsema's lifting: (1) \emph{Instruction 
lifting}: McSema 
while lifting uses code templates to generate IR sequences for each instruction. 
The injected bug forces the tool to choose wrong templates. The injected bug is 
targeted to affect the translation of $491$ unique instruction mnemonics that 
we collected from the compiled binaries of our evaluation test-suite.   
%
(2) \emph{Inferring data-section access constant}: \mcsema uses information from 
IDA~\cite{IDA} to know if an immediate operand used in data-section access 
instructions is a constant or a memory address. The introduced bug forces 
\mcsema to take the wrong decision.
%
(3) \emph{Maintaining 
correct data dependence among instructions}: The injected bug changes the 
order in which instructions are lifted, causing
data dependences between the dependent instructions to be violated
%
(4) \emph{Correctness of hoisting address computation code}: 
As mentioned earlier, 
\mcsema hoists the address computations of simulated registers back to the function
entry block, to be reused by 
later instructions throughout the function. Our injected bug forces 
the use of incorrect simulated address for general purpose registers and flags.

Each of the above bugs are injected one at a time and in combination and the 
resulting buggy lifter is tested against the \compd on the same evaluation 
test-suite before. All the injected bugs are correctly detected by the Matcher,
showing that \plv is useful for detecting bugs, not just proving correctness.

Note that only the first of these bugs would be caught by \siv: the binary instruction
semantics would not match with the LLVM IR sequence semantics in that case.
The second case would (in general) produce equivalent semantics between the X86 and
the LLVM IR sequence, and so \siv would not detect the bug.
The third and fourth bugs inherently span multiple instructions, and generate 
wrong code even if the individual instructions are translated correctly, so \siv 
would \emph{not} be able to detect them.

\section{Discussion}\label{sec:discussion}

In this section we discuss some limitations of our work and avenues for future
work.

\paragraph{Incomplete LLVM Semantics} The \LLVM semantics~\cite{LLVMSEMA} is
currently under development and does not support all LLVM abstractions, e.g.,
vector and floating point types and their associated operations, and various
intrinsics functions at the time of writing the paper. This is a limitation of
existing semantics and we believe the verification of lifted instructions that
use such unsupported features will work out-of-the-box when semantics are
available.

\paragraph{Formally Verified Normalizer} Our current implementation of the
normalizer uses a small number (of 15) LLVM passes to improve syntactic
matching between the McSema generated \s{T} and \s{T$^\prime$} proposed by
\compd. To prove soundness, these passes need to be formally verified to
perform only semantic preserving transformations. An alternative, more
promising approach is to develop simple graph transformations on SSA graphs to
mimic the transformations of LLVM passes and formally prove the
transformations preserve program semantics. We leave this to future work.

\paragraph{Extending to Other Lifters} Our current work focuses on McSema, the
most mature, open-source, binary to LLVM IR lifter. However, there are a
plethora of other lifters that are not formally verified. Extending our work
to support these systems is important for two reasons: (i) improving the
trust in binary lifters, and (ii) the improvements made to our system would
make it more generic enough for future binary lifters to get validation for
(nearly) free. We believe that this is mainly engineering effort that involves
customization of \compd to capture the idiosyncrasies of various lifters.

%Following are our current limitations.
%\begin{itemize}
    %\item 
    
    %\item The normalizer we are using is a heavyweight sequence of 
    %production compiler optimization passes. It is difficult to get confidence 
    %in their soundness, and hence the key weak link in the current approach.  
    %We are currently working to narrow down the list of optimizations we need 
    %to reduce the trust-base.  A more promising approach, for future,  would be 
    %to implement simpler  graph transformations on the SSA graphs being matched 
    %in order to mimic what the minimal LLVM passes do. We may even be able to 
    %write those as provably sound primitives using an interactive theorem 
    %prover, like Coq~\cite{Coq}.
    
    
%\end{itemize} 

\section{Related Work}\label{sec:RW}

Traditionally, \tv~\cite{Pnueli:1998} uses compiler instrumentation to help
generate a simulation relation to prove the correctness of compiler
optimizations~\cite{Rival:2004,Kanade:2006}. In our initial attempt to solve the
problem of \tv of the lifting of \ISA program, we tried to borrow insights
from such efforts. However, to be effective,  we believe our validator should
not instrument the lifter mainly because most the available lifters, being in
early development phase,  are updated and improved at a frantic pace. Without 
instrumentation, such simulation relations can be inferred by collecting
constraints from the input and output programs (as demonstrated in Necula's 
work~\cite{Necula:2000}). However, in the context of
\tv of binary lifting, such inference is not straight-forward  mainly because
the two program (\ISA binary and lifted IR) are structurally very different
with potentially different number of basic blocks  
% 
\footnote{Instructions like \instr{adcq} exhibit different semantic behavior
    based on input values and hence generate additional basic blocks upon 
    lifting,
    which are not explicit in the binary program}.
%
Unsurprisingly, a similar challenge poses a hard requirement of branch 
equivalence in Necula's approach.  
%
% is the same limitation that Necula~\cite{Necula:2000}
%poses a hard assumption of 
% pointed out in
%their paper. 
Consequently, we decided to move away from simulation-based
validation approaches.

 All the previous efforts, establishing the faithfulness of the binary 
 lifters,  can be broadly categorized to be based on (1)
  Testing, or (2) Formal Methods.

\subsection{Testing based Approaches}
This approach is similar to black-box testing in software engineering. Most
notable work include Martignoni et
al.~\cite{Martignoni:ISSTA2009, Martignoni:ISSTA2010,Martignoni:ASPLOS2012} and
Chen \etal~\cite{CLSS2015}.


Martignoni et al.~\cite{Martignoni:ISSTA2009, Martignoni:ISSTA2010} proposes
hardware-cosimulation based testing on QEMU~\cite{QEMU:USENIX05} and
Bochs~\cite{Bochs1996}.  Specifically, they compared the state between actual
CPU and  IA-32 CPU emulator (under test) after executing randomly selected
test-inputs on randomly chosen instructions  to discover any semantic
deviations. Although, a simple and scalable approach, it's effectiveness is
limited because many semantics bugs in binary lifters are triggered upon a
specific input and exercising all such corner inputs, using randomly generated
test-cases, is impractical.

%%architecture-state-comparison problems,

In general, such testing based approaches are unsuitable for validating whole
program (or even basic block) translations because even a correctly translated
program may not always produce exactly the same output as the original program
due to the differences in modeling of the architectural states in the
translated program (or basic block) vs the original program. Although, it is 
possible to engineer out such architecture-state-comparison problems, but still 
these approaches might
not
detect some intermediate mistranslated instructions which are course corrected
at the end. As an example, suppose  a register is assigned values twice in a
program and the first assignment is mistranslated but the second assignment
statement is translated correctly. Comparing the architecture states at the end
of the program (or basic block) may not discover the mis-translation.

%%

Chen \etal~\cite{CLSS2015} proposed validating the static binary translator
LLBT~\cite{LLBT2012} and the hybrid binary translator~\cite{LLVMDBT2012},
  re-targeting ARM programs to x86 programs. First, an ARM program is
  translated offline to x86 program (via an intermediate translation to LLVM 
  IR). Next, the translated x86 binary is
  executed  directly on a x86 system while the original ARM binary runs on the
  QEMU emulator. During run time, both the ARM binary and the translated x86
  binary produce a sequence of  architecture states, which are compared at the
  granularity of single instruction after solving the 
  architecture-state-comparison problem, as mentioned above. The validator is
  evaluated using the ARM code compiled from EEMBC 1.1 benchmark. Like previous
  approach, the validation of single instruction's translation is
based on testing and hence shares the same limitation of not being exhaustive.
%%

Martignoni \etal~\cite{Martignoni:ASPLOS2012} applied symbolic execution on a
Hi-Fi (``faithful and more complete in terms of IA-32 ISA'')
emulator\cite{Bochs1996}'s implementation of an instruction semantics to
generate
high-fidelity test-inputs to validate a ``buggier and less complete'' Lo-Fi
emulator~\cite{QEMU:USENIX05},
by executing the binary instruction twice, once on a real hardware and next on
the Lo-Fi emulator, and the output states are matched. However, the 
work~\cite{Martignoni:ASPLOS2012} does not aim to validate the
translation of \ISA programs,
which is one of out primary contributions.

Note that an approach as above cannot scale naturally to binary function  
validation; A set of high-coverage test-inputs for all the constituent 
instructions of a function cannot trivially derive  high-coverage test-inputs 
for the whole function. 
%This is mainly mainly because  such test-inputs are generated in an isolated 
%context of that particular instruction and may not even be satisfiable in the 
%whole program context. 

%For each path explored during the symbolic execution of an instruction's
%implementation, the underlying decision procedure computes an assignment of
%bits to the symbolic inputs states that would cause the emulator to execute
%that path: such assignments, serve as the test-inputs, together cover
%all
%behaviors of that instruction.
%However, if the goal is to validate whole-program translation and if
%we have
%such test-inputs for  all the constituent instructions of the
%program, still it will not help to generate high coverage test-inputs for the
%whole program mainly because  such test-inputs are generated in an isolated
%context and may not even be satisfiable in the whole program context.

%%


%%
%Note that, even though Martignoni \etal~\cite{Martignoni:ASPLOS2012}
%symbolically explored the test-cases which is supposed to cover all the paths
%of a given instruction's implementation, but being a differential testing-based
%approach, the faithfulness depends directly on  the fidelity of the Hi-Fi
%emulator. A wrong implementation or omission of a particular switch case of
%instruction semantics in the Hi-Fi, will lead to test-cases insufficient to
%explore all the paths and hence find bugs in the Low-Fi emulator. Also, the
%method can capture  deviations in the behavior of only those
%instructions which are implemented in both the emulators.\footnote{We
%  note that our proposed semantics-driven \tv approach shares similar
%    assumptions about the faithfulness of the semantics.}.
%%
%Moreover, the symbolic execution of an instruction's implementation in the
%Hi-Fi emulator is achieved using an X86 interpreter FuzzBALL. A bug in the
%interpreter will affect the generation of high-fidelity test cases for a
%particular instruction, leading to incomplete coverage of that instruction's
%implementation in Low-Fi emulator.
%%
%However, their approach does not consider the floating point instruction
%  because the employed symbolic execution engine (FuzzBALL) does not support
%    it.

%    Schwartz \etal~\cite{Schwartz:2013} proposed control flow structure recovery by
%    employing semantics preventing schema and tested their binary to C decompiler,
%    Phoenix, which is based on BAP~\cite{BAP:CAV11}, on a set of 107 real
%    world programs from GNU coreutils. Along similar lines,
%    %
%    Yakdan \etal~\cite{Yakdan2015NDSS} presented a decompiler, DREAM, to offer a
%    goto-free output. DREAM uses a novel pattern independent control-flow
%    structuring algorithm that can recover all control constructs in binary
%    programs and produce structured decompiled code without any goto statement. The
%    correctness of our algorithms is demonstrated using the GNU coreutils suite of
%    utilities as a benchmark.
%
%    Andriesse \etal~\cite{nucleus2017EuroSP} proposes a function detection
%    algorithm, Nucleus, for binaries. The algorithm does not require function
%    signature information or any learning phase. They evaluated Nucleus on a
%    diverse set of $476$ C and C++ binaries, compiled with gcc, clang and Visual
%    Studio for x86 and x64, at optimization levels O0--O3.
%
%    Martignoni et al.~\cite{Martignoni:ISSTA2009, Martignoni:ISSTA2010} attempted
%    to leverage differential testing on QEMU~\cite{QEMU:USENIX05} and
%    Bochs~\cite{Bochs1996}. Particularly, they compared the state between a
%    physical and an emulated CPU after executing randomly chosen instructions on
%    both to discover any semantic deviations. Although their technique can be
%    applied to testing binary lifters, it is fundamentally limited because its
%    effectiveness largely depends on randomly generated test cases. Typically,
%    semantic bugs in binary lifters are triggered only with specific
%    operand values. Therefore, a random test case generation does not
%    help much in finding such bugs.

\subsection{Formal Methods based Approaches}
Followings are the effort to establish strong guarantees for binary
translations using formal methods.

MeanDiff~\cite{ASE2017} proposed an N-version IR testing to validate three
binary lifters, BAP~\cite{BAP:CAV11}, BINSEC~\cite{BINSEC2011}, and
PyVEX~\cite{PYVEX} by comparing their translation of a single binary
instruction to BIL, DBA, and VEX IRs respectively. The individual IRs are then
converted to common IR representations which are then symbolically executed to
generate symbolic summaries for comparison using a SAT solver.  The above
approach shares the same fundamental limitations of any differential testing
techniques. For example, if all the binary lifters are in  sync
on the behavior of a particular instruction, we get more confidence in correct
 implementation of that instruction's semantics in all of them, but we
cannot rule
out the possibility of all being incorrect. Also, even if there a disagreement
in the behavior of two or more translators, still it might just be a false
alarm
in case all the candidates are buggy. However, the work~\cite{ASE2017}
is primarily focused on validating single
instruction and the problem
of handling multiple instruction is left as future work.

%%

Moreover, as candidly mentioned in the paper~\cite{ASE2017}, one of the 
motivations for
relying on differential testing  is that there were no
formal specification of \ISA ISA at the time of
writing the paper. Whereas we do not have such limitation because of the formal
\ISA ISA specification~\cite{DasguptaAdve:PLDI19} made public recently.
Empowered with that, we can build a symbolic formula that encodes all execution
paths of an IR instance lifted from a single machine instruction and then
check if the symbolic formula matches the formal specification of the
instruction, which is exactly what we did in this work.

%%
The work closest to ours, in term of the goals, is the translation verifier,
Reopt-vcg~\cite{Galois:SPISA19}, which is developed to cater the verification
challenges specific to the translator Reopt~\cite{reopt}. The verifier, which
validates the translations at basic-block level, is assisted by various
manually written annotations, which are prone to errors. Such annotations
could have been generated by instrumenting the lifter. 
%
 Contrary to that, our approach  does not need any such 
instrumentations, thereby avoided
the overhead of maintaining instrumentation patches whenever the lifter
design/implementation is updated. 
Moreover, 
the validator uses the semantics
definitions of a small subset of \ISA, which in turn limits its 
applicability
to small programs. We avoided this
limitation by incorporating the most complete and heavily tested \ISA
semantics~\cite{DasguptaAdve:PLDI19} available to date.

%
%More importantly, we aim for whole program
%validation.  
%
%
%Reopt-vcg~\cite{Galois:SPISA19} is closest to ours in terms of its goal of
%proving that a translated LLVM program is a refinement of a \ISA program.  The
%translation verifier takes a binary executable, LLVM bitcode file,
%             manually provided annotations  along with several simplifying 
%            assumptions specific to the translator (Reopt~\cite{reopt}) to 
%            generates proof 
%            obligations in SMTLIB to
%            verify each basic block independently. The annotation generation 
%            could have been automated by instrumenting the lifter,  which adds 
%            both implementation and maintenance overheads. 
% %           First, the approach
% %           is aimed for a particular decompiler~\cite{reopt} and generalizing
% %           it to others needs substantial instrumentation to correctly emit 
% %the
%  %          annotations. 
%  Contrary to that, our approach  does not need any such 
%            instrumentations and provides a simpler approach for program  level 
%            validation. 
%
        
  
%\cmt{The annotations are 
%    used to
%    identify basic block correspondence, define the
%    appropriate pre-conditions for the block, distinguish between 
%    operations with and without side effects, and identify argument 
%    mapping to
%    machine code registers.}        
  
        %            \todo[inline,color=yellow]{Second, the 
        %            verifier use a small subset of manual written \ISA 
        %semantics which 
        %            limits its applicability to small program. On the 
        %contrary, our 
        %            \siv is based on the most complete and heavily tested 
        %formal 
        %            semantics available to date and hence applicable to 
        %real-world 
        %            programs.}
%%
% \cmt{The underlying comparison tool uses 
%    those 
%    annotations to step through both the machine code and LLVM in 
%    parallel and ensure that for which each LLVM operation with side 
%    effects, including memory reads and writes, has an 
%    equivalent write in the machine code.}

%\subsection{Using Machine Learning} Another recent work by Schulte
%\etal~\cite{eschulte2018bed} proposed Byte-Equivalent Decompilation (BED) which
%leveraged a genetic optimization algorithm to infer C source code from a
%binary. Given a target binary and an initial population of C code as
%decompilation candidates, they  drive a genetic algorithm to improve the
%initial candidates, driving them closer (using compilation to binary) to
%byte-equivalence w.r.t the target binary. The byte equivalence  is simply the
%edit distance to the target binary. As hypothesized in the future work section
%of the paper~\cite{eschulte2018bed}, BED could be applied to LLVM IR instead of
%C to evolve lifting from machine code to LLVM IR and may work well due to the
%relative simplicity of LLVM IR as compared to the C. Being byte-equivalent, the
%generated LLVM IR will be the faithful evolution from the machine code.
%However, as shown in the paper, this approach worked moderately well for
%smallish binaries. For example, out of $22$ binaries under test, only $4$
%achieve full byte equivalence when the initial population is augmented with
%decompilation candidates from the HEX-RAYS~\cite{hexray} Decompiler. It is
%still an open problem to realized an end-end byte-equivalent binary to LLVM
%decompiler using purely genetic optimization algorithm.  \cmt{only $3$ achieve
%  full byte equivalence when the initial population does not include decompiler
%    seeds, and}


%\cmt{
%    Myreen et al.~\cite{Myreen:FMCAD:2008,Myreen:FMCAD:2012} proposed
%    ``decompilation into logic'' which, given some concrete machine code and a
%    model of an ISA, extracts logic functions or symbolic summaries which
%    captures
%    the functional behavior of the machine code. The decompiler works on top of
%    ISA
%    models for IA-32 \cite{Karl2003}, ARM~\cite{Fox2003} and
%    PowerPC~\cite{Leroy:2006}. Assuming that the ISA models are trusted, the
%    extracted functions can be used to prove properties of the original machine
%    code. However, the work has not been applied to validate the binary
%    translation
%    to an IR.  A recent work by Roessle et al.~\cite{Roessle:CPP19} improves the
%    aforementioned idea  by including a subset of \ISA, derived mostly from
%    Strata~\cite{Heule2016a}, in the trust-base of ISA models.
%}
%
%
%%%
%
%%

%\cmt{Myreen et al.~\cite{Myreen:FMCAD:2008,Myreen:FMCAD:2012} extracted
%    function-level symbolic summaries which indeed is a promising building block
%    towards establishing correctness of binary lifters.\cmt{, which, however,
%        has many additional challenges to deal with (Refer
%        Section~\ref{sec:challenges}). Moreover,} However, both Myreen et al.
%    and Roessle et al. have limited \ISA instruction set coverage, which
%    might restricts their application on many  real-world binaries.}
%
%\cmt{Being a differential testing method, only those instructions which are
%    supported in all the translators can be validated with higher confidence
%    than
%    the ones which are not supported in one or more.
%
%    Also, as MeanDiff is testing
%    multiple binary lifters together, hence it cannot be used to establish the
%    faithfulness in lifting the semantics of an instruction which is not
%    implemented in any one of them.
%
%    In this approach of differential testing, whether a particular instruction
%    is
%    going to be validated depends on its availability in other translators,
%    even if
%    some translator as a much better instruction support. In our case, we are
%    testing McSema against the most complete user level \ISA ISA
%}
%\cmt{MeanDiff neither
%    handle floating point operations, nor the instructions which does not
%    manifest
%    their side-effects (like flag updates) explicitly.  Moreover, MeanDiff
%    reports
%    a bug whenever a deviation is detected w.r.t the
%    instruction-semantics-behavior
%    in at least two binary lifters. But even if all the binary lifters are in
%    sync
%    on the behavior of a particular instruction, we cannot guarantee that all
%    the
%    lifters are faithful in lifting that instruction, which is however, a
%    general
%    limitation of differential testing based approach. Also, as MeanDiff is
%    testing
%    multiple binary lifters together, hence it cannot be used to establish the
%    faithfulness in lifting the semantics of an instruction which is not
%    implemented in any one of them.
%}

%\section{Conclusion and Future Work}\label{sec:conc}
\section{Conclusion}\label{sec:conc}
We have presented the most complete formal semantics of \ISA user-level instructions
to date, and have thoroughly tested it using synthesized test inputs and the GCC torture tests.
%in different formal analyses such as symbolic execution, deductive verification, and translation validation.
We have also illustrated several potential uses of the semantics which are realized by the formal analysis tools
derived right from the \K specification. 
%Using the \K framework for our specification automatically provides several formal analysis tools from the specification, which greatly simplified these applications.
The \K framework also enables us to represent a semantics as SMT theories,
which other projects can
leverage for their own purposes.
%We describe several practical lessons we have learned from our experience in developing the semantics, which could be useful for future formal specifications of processor ISAs.


%As mentioned in Section~\ref{sec:Approach:Overview}, we found that our ideas of extending \Strata do not scale because we need information about the instructions, whose semantics we want to learn, in order to constraint the search space. This lesson suggested a new idea: the information needed to reduce the search space can be automatically extracted from the manual. This  information does not need to be precise, and we believe that such rough information can be automatically extracted from the manual using text processing.  We plan to explore this idea while defining the semantics of unspecified and/or new instructions.

%\SC{We plan to test our model against various x86-64 implementations (like AMD), which could uncover flaws in those implementations and/or additional imperfections in the manual.}

%\SC{Also, we aim to use our semantics for translation validation of the entire LLVM compiler back-end for X86-64, i.e., hundreds of thousands of lines of C++ code. Scalability is achieved by the modularity of the translation validation technique, where the verification of (small) sub-components are verified individually (hence can be massively parallelized) and their results are combined to obtain the final claim about the entire system.}

 

%\K  has success stories about defining the formal semantics of production languages like C~\cite{KC} and LLVM~\cite{KLLVM}. We can use that towards the benefit of binary lifters like ~\cite{McSema:Recon14, FCD} which lifts the binary code to LLVM IR for binary analysis and/or optimization. Having the semantics of both the source (\ISA) and the target language(LLVM) help in verifying the translation using symbolic reasoning and hence enhance trust in  the translation.

%\textit{Acknowledgements.}
%We warmly thank the \K team
%   for their technical support throughout the project.  Also we thank the Strata
%   developers for promptly confirming
%   our reported bugs and for answering all our questions in great detail.
   

