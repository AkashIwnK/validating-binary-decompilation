\section{Discussion}\label{sec:discussion}

In this section we discuss some limitations of our work and avenues for future
work.

\paragraph{Incomplete LLVM Semantics} The \LLVM semantics~\cite{LLVMSEMA} is
currently under development and does not support all LLVM abstractions, e.g.,
vector and floating point types and their associated operations, and various
intrinsics functions at the time of implementation. This is a limitation of
existing semantics and we believe the verification of lifted instructions that
use such unsupported features will work out-of-the-box when semantics are
available.

\paragraph{Formally Verified Normalizer} Our current implementation of the
normalizer uses a small number (of 15) LLVM passes to improve syntactic
matching between the McSema generated \s{T} and \s{T$^\prime$} proposed by
\compd. To prove soundness, these passes need to be formally verified to
perform only semantic preserving transformations. An alternative, more
promising approach is to develop simple graph transformations on SSA graphs to
mimic the transformations of LLVM passes and formally prove the
transformations preserve program semantics. We leave this to future work.

\paragraph{Extending to Other Lifters} Our current work focuses on McSema, the
most mature, open-source, binary to LLVM IR lifter. However, there are a
plethora of other lifters that are not formally verified. Extending our work
to support these systems is important for two reasons: (i) improving the
trust in binary lifters, and (ii) the improvements made to our system would
make it more generic enough for future binary lifters to get validation for
(nearly) free. We believe that this is mainly engineering effort that involves
customization of \compd to capture the idiosyncrasies of various lifters.

%Following are our current limitations.
%\begin{itemize}
    %\item 
    
    %\item The normalizer we are using is a heavyweight sequence of 
    %production compiler optimization passes. It is difficult to get confidence 
    %in their soundness, and hence the key weak link in the current approach.  
    %We are currently working to narrow down the list of optimizations we need 
    %to reduce the trust-base.  A more promising approach, for future,  would be 
    %to implement simpler  graph transformations on the SSA graphs being matched 
    %in order to mimic what the minimal LLVM passes do. We may even be able to 
    %write those as provably sound primitives using an interactive theorem 
    %prover, like Coq~\cite{Coq}.
    
    
%\end{itemize} 
