%% Custom Packages
\usepackage[T1]{fontenc}
\usepackage{listings}
\usepackage{color}
\usepackage{xspace}
\usepackage[title]{appendix}
%\usepackage{tikz}
%\usepackage{pgf-pie}
\usepackage{forest}
\usepackage{amsmath,amssymb}
\usepackage{ctable}
\usepackage{pifont}
\usepackage{calculator}



% FIXME: overleaf broken if uncommented
% \usepackage{tikz-qtree}
% \usetikzlibrary{arrows,shapes,positioning,shadows,trees}
% \usetikzlibrary{shadows,trees}


%% Custom Commands
\def\Code#1{\texttt{#1} }
%\def\Comment#1{}
\def\Comment#1{\textbf{\textsl{\color{red}  $\langle\!\langle$#1$\rangle\!\rangle$}} }
\newcommand{\percentage}[2]{\DIVIDE{#1}{#2}{\duv}\MULTIPLY{\div}{100}{\res}$\res\%$}
\newcommand{\revisit}[1]{{\color{red} Sandeep: #1}}
\newcommand{\Qd}[1]{{\color{red} Daejun: #1}}
\newcommand{\Qt}[1]{{\color{red} Theo: #1}}
\newcommand{\BW}[1]{{\color{red} Borrowed: #1}}
\newcommand{\Added}[1]{{\color{red} #1}}
\newcommand{\cmt}[1]{}
\newcommand{\xmark}{{\color{red} \ding{55}}}
\newcommand{\cmark}{{\color{green} \ding{51}}}
\newcommand{\ISA}{x86-64\xspace}
\newcommand{\LLVM}{LLVM IR\xspace}
%\newcommand{\K}{\mbox{$\mathbb{K}$}\xspace}
\newcommand{\Z}{$\mathbb{Z}3$\xspace}
\newcommand{\uif}{uninterpreted functions}
\newcommand{\Strata}{Strata\xspace}
\newcommand{\Stoke}{Stoke\xspace}
\newcommand{\initS}{{\tt initial search}}
\newcommand{\secS}{{\tt secondary searches}}
%\newcommand{\K}{\ensuremath{\mathcal{{\tt K}}}\xspace}
\newcommand{\TS}[1]{{\tt #1}}
%\newcommand{\instr}[1]{\texttt{#1}}
\newcommand{\instr}[1]{\textbf{\color{brown}\m{#1}}}
\newcommand{\reg}[1]{\s{\%#1}}
\newcommand{\mem}[2]{\s{#1(\%#2)}}
\newcommand{\opcode}[1]{\ensuremath{#1}}
%\newcommand{\cond}[1]{\ensuremath{#1}}
\newcommand{\extract}{\emph{extract}\xspace}
\newcommand{\extractMInt}{\emph{extractMInt}\xspace}
\newcommand{\false}{\textbf{False}}
\newcommand{\true}{\textbf{True}}
\newcommand{\bool}{\texttt{Bool}\xspace}
\newcommand{\incfig}[1]{\includegraphics[scale=.7]{#1}}
\newcommand{\CF}[2]{$\s{F}_{\s{#2}}^{\s{#1}}$}
\newcommand{\GN}[2]{$G{[#2]}^{#1}$}
\newcommand{\udef}{\emph{undef}\xspace}
\newcommand{\bv}[2]{$#1\text{'}#2$\xspace}

\newcommand{\rating}[1]{%
    \begin{tikzpicture}[x=1ex,y=1ex]
    \begin{scope}
    \clip (0,1) circle (1);
    \fill[black] (-1,0) rectangle (1,#1/50);
    \end{scope}
    \draw[black, thin, radius=1] (0,1) circle;
    \end{tikzpicture}%
}

% Current Support
%\newcommand{\currentIS}{$3186$} %including jmp label
\newcommand{\currentIS}{$3155$}
\newcommand{\currentIntel}{$774$}
%\newcommand{\currentManual}{$1199$} %3104 - 1905
%\newcommand{\currentManual}{$1281$} %3186 - 1905
%\newcommand{\currentManualPerc}{$40\%$} %100 - \strataPerc
% Total
\newcommand{\totalIS}{$3736$}
\newcommand{\totalIntel}{$996$}
%\newcommand{\totalIntel}{$1,000$}
\newcommand{\dup}{$109$}
% Strata
\newcommand{\strataIS}{$1796$}
\newcommand{\strataIntel}{$466$}
\newcommand{\strataWithDupIS}{$1905$}
\newcommand{\strataRegVarIS}{$692$}
% Unsupported
\newcommand{\system}{$210$}
\newcommand{\Xmmx}{$336$}
\newcommand{\crypto}{$35$}

\newcommand{\strataPerc}{$47\%$} % 466/996 or  1905 / 3736
\newcommand{\goelPerc}{$33\%$} 
% Stoke disjoin from Strata
%\newcommand{\stokeIS}{$332$} % 262 + 15 + 9 + 46. ALso 332/3767 == 9%
% 1432(strata common) + 332
\newcommand{\stokeIS}{${\sim}1764$}
%\newcommand{\stokeExcPerc}{$9\%$}
% Strata stoke combined
%\newcommand{\strataPlusStokeIS}{$2237$} % 

%\newcommand{\unsupp}{$939$}

%%%%%% Immediates
%\newcommand{\ImmUg}{$146$} % 118 + 28
%\newcommand{\ImmTotal}{$308$}
%\newcommand{\ImmG}{$190$}

%%%%%%% Registers
%\newcommand{\RegTOTAL}{$1133$} % 1083 + 50
%\newcommand{\RegSTRAT}{$742$} % 692 + 50
%\newcommand{\RegSTOK}{$262$}
%\newcommand{\RegMAN}{$129$}

%%% toture status
\newcommand{\TortureTotal}{$1576$} %
\newcommand{\TortureExclude}{$6$} % 6 + 22
\newcommand{\TortureInclude}{$1548$}
\newcommand{\TortureUifsInstr}{$293$} % 134(all three jobs) +  48
\newcommand{\TortureUifs}{$35$}
\newcommand{\TortureCoverage}{$963$}
%%% Undef counts
\newcommand{\undefTotal}{$474$}
\newcommand{\undefIntel}{$32$}
\newcommand{\undefPerc}{$3$} %32/1000

\PassOptionsToPackage{pdftex,usenames,dvipsnames,svgnames,x11names}{xcolor}
\PassOptionsToPackage{pdftex}{hyperref}
\usepackage[style=math]{k}


% Slightli modified original version of \reduce. Altered baseline for more compactness.
% No support for multiline.
\newcommand{\reduceClassic}[2]{\hbox{%
  \begin{tikzpicture}[baseline=(top.south), %(top.base), - default, less compact
                      inner xsep=0pt,
                      inner ysep=.3333ex,
                      minimum width=2em]
    \path
          % Original version. No support for line wrapping.
          node (top) [inner ysep=1ex]{$#1$ \mathstrut}

          % New version. Line wrapping support.
          %node (top) [inner ysep=1ex]{$ \begin{array}{@{}c@{}} #1 \end{array} $ \mathstrut}
          (top.south)
          % Original version. No support for line wrapping.
          node (bottom) [anchor=north, inner ysep=.5ex] {$#2$};

          % New version. Line wrapping support.
          % Adds a little bit of vertical space, but the difference is truly insignificant. All the experiments below failed to remove it.
          %node (bottom) [anchor=north, inner ysep=.5ex] {$ \begin{array}{@{}c@{}} #2 \end{array} $};
          % no extra effect
          % node (bottom) [anchor=north, inner ysep=.5ex] {\vspace{-1em} $ \begin{array}{@{}c@{}} #2 \end{array} $};
          % trying mathstrut - some horizontal re-alignment, but no vertical
          % node (bottom) [anchor=north, inner ysep=.5ex] {$ \begin{array}{@{}c@{}} #2 \end{array} $ \mathstrut};
          % no outer ysep (if no inner - looks bad)
          %node (bottom) [anchor=north, inner ysep=.5ex, outer ysep=0] {$ \begin{array}{@{}c@{}} #2 \end{array} $};
          % \vskip -1em just don't compile no matter where we put it
    \path[draw,thin,solid] let \p1 = (current bounding box.west),
                               \p2 = (current bounding box.east),
                               \p3 = (top.south)
                           in (\x1,\y3) -- (\x2,\y3);
    % Solid arrow (augmenting the solid line).
    \path[fill] (top.south) ++(2pt,0) -- ++(-4pt,0) -- ++(2pt,-1.5pt) -- cycle;
  \end{tikzpicture}%
}}

% Defalut version of \reduce in this document.
%   Support for multi-line LHS and RHS
%   Good compactness. Separators adjusted to be aligned with \reduceClassic
\newcommand{\reduceMulti}[2]{\hbox{%
  \begin{tikzpicture}[baseline=(top.south), %(top.base), - default, less compact
                      inner xsep=0pt,
                      inner ysep=.3333ex,
                      minimum width=2em]
    \path
          % New version. Line wrapping support.
          node (top) [
            %inner ysep=1ex
            inner ysep=0.6ex
          ]{ $ \begin{array}{@{}c@{}}
                #1
               \end{array} $ \mathstrut}
          (top.south)
          % New version. Line wrapping support.
          node (bottom) [
            anchor=north,
            %inner ysep=.5ex
          ] {
            $ \begin{array}{@{}c@{}}
              #2
            \end{array} $};
    \path[draw,thin,solid] let \p1 = (current bounding box.west),
                               \p2 = (current bounding box.east),
                               \p3 = (top.south)
                           in (\x1,\y3) -- (\x2,\y3);
    % Solid arrow (augmenting the solid line).
    \path[fill] (top.south) ++(2pt,0) -- ++(-4pt,0) -- ++(2pt,-1.5pt) -- cycle;
  \end{tikzpicture}%
}}

%Special version of \reduce with modified baseline, for better rendering of multiline
% LHS and RHS
\newcommand{\reduceCompact}[2]{\hbox{%
  \begin{tikzpicture}[baseline=(bottom), %(top.base), - default, less compact
                      inner xsep=0pt,
                      inner ysep=.3333ex,
                      minimum width=2em]
    \path
          node (top) [inner ysep=0.6ex]{$ \begin{array}{@{}c@{}} #1 \end{array} $ \mathstrut}
          (top.south)
          node (bottom) [anchor=north] {$ \begin{array}{@{}c@{}} #2 \end{array} $};
    \path[draw,thin,solid] let \p1 = (current bounding box.west),
                               \p2 = (current bounding box.east),
                               \p3 = (top.south)
                           in (\x1,\y3) -- (\x2,\y3);
    % Solid arrow (augmenting the solid line).
    \path[fill] (top.south) ++(2pt,0) -- ++(-4pt,0) -- ++(2pt,-1.5pt) -- cycle;
  \end{tikzpicture}%
}}

%\renewcommand{\reduce}[2]{\reduceClassic{#1}{#2}}
\renewcommand{\reduce}[2]{\reduceMulti{#1}{#2}}




%\lstset{captionpos=t,tabsize=3,frame=no,keywordstyle=\color{blue},
%        commentstyle=\color{gray},stringstyle=\color{red},
%        breaklines=true,showstringspaces=false,emph={label},
%        basicstyle=\ttfamily}

% Required in order to make \kall cells inside comments black.
\renewcommand{\kall}[3][black]{\mall{#1}{#2}{#3}}

% Environment "kdefinition" has effect only in poster style, thus in math style may be safely deleted.

%Continuation of a syntax definition on a new line
\newcommand{\syntaxContNewLine}[3][\defSort]{\par\indent\rulebox{%
  $\setlength{\syntaxlength}{\widthof{$\mathrel{::=}$}}%
  \setlength{\syntaxlength}{.5\syntaxlength}%
  \addtolength{\syntaxlength}{\widthof{\syntaxKeyword$#1$}}%
  \hspace{\syntaxlength}%
  \;\;\;\;\;\;\;{#2}$ \ifthenelse{\equal{#3}{}}{}{[#3]}%
  }%\k@markPosition%
}

%Should be put after a syntaxLong.
\newcommand{\syntaxEnd}[3][\defSort]{
  \indent\rulebox{%
  $\setlength{\syntaxlength}{\widthof{$\mathrel{::=}$}}%
  \setlength{\syntaxlength}{.5\syntaxlength}%
  \addtolength{\syntaxlength}{\widthof{\syntaxKeyword$#1$}}%
  \hspace{\syntaxlength}$%
  }%\k@markPosition%
}

\newcommand{\syntaxLong}[3][\defSort]{\rulebox{%
\syntaxKeyword
$
  \begin{array}[t]{@{}l@{}}
  #1 \\
  \mathrel{::=}{#2}
  \end{array}
$ {}%
}%\k@markPosition%
}

% Grigore's idea macro
\newcommand{\idea}[1]{
  \begin{quote}
    \rule{.45\textwidth}{.5pt}\newline
    {\em #1}
    \vspace*{-1ex}\newline \rule{.45\textwidth}{.5pt}
  \end{quote}
}

\newenvironment{ideas}
{ \begin{quote}
    \rule{.45\textwidth}{.5pt}
    \newline
    \begin{em}
} {
    \end{em}
    \vspace*{-1ex}
    \leavevmode
    \newline
    \rule{.45\textwidth}{.5pt}
  \end{quote}
}

%Enforcing black cells
\renewcommand{\kall}[3][white]{\mall{black}{#2}{#3}}
\renewcommand{\kallLarge}[3][white]{\mallLarge{black}{#2}{#3}}
\renewcommand{\kprefix}[3][white]{\mprefix{black}{#2}{#3}}
\renewcommand{\ksuffix}[3][white]{\msuffix{black}{#2}{#3}}
\renewcommand{\kmiddle}[3][white]{\mmiddle{black}{#2}{#3}}

% Settigns required for Chucky's background section
\usepackage{acronym}

\providecommand{\Sec}{}
\renewcommand{\Sec}{Section~}
\newcommand{\Fig}{Figure~}

\newcommand{\cellname}[1]{\textsf{#1}}

\newcommand{\kequation}[2]{\begin{equation*}{\small#2}\end{equation*}}

%Probably a mapsto with spacing
\newcommand{\mapstox}{\small\mathrel{\mapsto}}

%For spacing between cell lines
%\newcommand{\kBR}{\\[0.3em]}

% General
\newcommand{\w}[1]{\ensuremath{\textit{#1}}}
\newcommand{\m}[1]{\ensuremath{\texttt{#1}}}
\newcommand{\s}[1]{\ensuremath{\textsf{#1}}}
\newcommand{\p}[1]{\ensuremath{\left(#1\right)}}
\newcommand{\pl}[1]{\ensuremath{\left\langle#1\right\rangle}}
\newcommand{\OR}{\mbox{ }|\mbox{ }}
\newcommand{\st}{.\mbox{ }}
\newcommand{\finto}{\ensuremath{\stackrel{\mathtt{fin}}{\longrightarrow}}}
\newcommand{\defeq}{\ensuremath{\stackrel{\mathtt{def}}{=}}}
\newcommand{\cond}[1]{\ensuremath{\left\{\begin{array}{ll} #1 \end{array}\right.}}
\newcommand{\lst}[1]{\begin{itemize} {#1} \end{itemize}}
\newcommand{\pby}[1]{\hspace*{\fill}{#1}}
\newcommand{\slide}[2][]{ \begin{frame} \frametitle{#1} {#2} \end{frame} }
\newcommand{\etal}{\textit{et~al.}\xspace}

% For references
\newcommand{\fig}[1]{Figure~\ref{#1}}
\newcommand{\lem}[1]{Lemma~\ref{#1}}
\newcommand{\theo}[1]{Theorem~\ref{#1}}
\newcommand{\coro}[1]{Corollary~\ref{#1}}
\newcommand{\defn}[1]{Definition~\ref{#1}}
\newcommand{\rmrk}[1]{Remark~\ref{#1}}
\newcommand{\exam}[1]{Example~\ref{#1}}
\newcommand{\sect}[1]{$\S$~\ref{#1}}

%\newcommand{\todo}[1]{}
%\newcommand{\todo}[1]{{\textcolor{red}{\textbf{[[{#1}]]}}}}

\usepackage{ucs} % for unicode characters (just for \Rosu and \Serbanuta)
\newcommand{\sh}{\unichar{0537}}
\newcommand{\Sh}{\unichar{0536}}
\PrerenderUnicode{\sh}
\PrerenderUnicode{\Sh}
\newcommand{\Rosu}{Ro{\sh}u\xspace}
\newcommand{\Stefanescu}{{\Sh}tef{\u a}nescu\xspace}

\newcommand{\JS}{JavaScript\xspace}
\newcommand{\ES}{ECMAScript\xspace}
%\newcommand{\K}{\ensuremath{\mathbb{K}}\xspace}
%\newcommand{\KJS}{\ensuremath{\mathbb{K}}JS\xspace}
\newcommand{\KJS}{KJS\xspace}
\newcommand{\LJS}{\ensuremath{\lambda_{\w{JS}}}\xspace}
\newcommand{\spec}{specification\xspace}

\lstdefinelanguage{JavaScript}{
  keywords={break, case, catch, continue, debugger, default, delete, do, else, finally, for, function, if, in, instanceof, new, return, switch, this, throw, try, typeof, var, void, while, with},
  morecomment=[l]{//},
  morecomment=[s]{/*}{*/},
  morestring=[b]',
  morestring=[b]",
  sensitive=true
}

\definecolor{orange}{rgb}{1,0.5,0}
\definecolor{darkgreen}{rgb}{0.0, 0.5, 0.0}
\newcommand{\note}[2]{\textbf{\textit{\textcolor{#1}{[[{#2}]]}}}}
\newcommand{\marker}[1]{} %{\note{orange}{{#1}}}
\newcommand{\daejun}[1]{\note{red}{Daejun: {#1}}}
\newcommand{\andrei}[1]{\note{darkgreen}{Andrei: {#1}}}
\newcommand{\grigore}[1]{\note{blue}{Grigore: {#1}}}



\definecolor{codegreen}{rgb}{0,0.6,0}
\definecolor{codegray}{rgb}{0.5,0.5,0.5}
\definecolor{codepurple}{rgb}{0.58,0,0.82}
\definecolor{backcolour}{rgb}{0.95,0.95,0.92}


\lstdefinestyle{Bash}{
    language=Bash,                % choose the language of the code
    basicstyle=\footnotesize,       % the size of the fonts that are used for the code
    numbers=left,                   % where to put the line-numbers
    numberstyle=\tiny\color{codegray},      % the size of the fonts that are used for the line-numbers
    stepnumber=1,                   % the step between two line-numbers. If it is 1 each line will be numbered
    numbersep=5pt,                  % how far the line-numbers are from the code
    backgroundcolor=\color{white},  % choose the background color. You must add \usepackage{color}
    showspaces=false,               % show spaces adding particular underscores
    showstringspaces=false,         % underline spaces within strings
    showtabs=false,                 % show tabs within strings adding particular underscores
    frame=single,           % adds a frame around the code
    %tabsize=2,          % sets default tabsize to 2 spaces
    captionpos=b,           % sets the caption-position to bottom
    breaklines=true,        % sets automatic line breaking
    breakatwhitespace=false,    % sets if automatic breaks should only happen at whitespace
    escapeinside={\%*}{*)},          % if you want to add a comment within your code
    commentstyle=\color{gray},
    keywordstyle=\color{blue},
    morekeywords={andnq, jp, jz, movw, movq, xorq, orq, retq, pushw}
}

\lstdefinestyle{C++}{
    language=C++,                % choose the language of the code
    basicstyle=\footnotesize,       % the size of the fonts that are used for the code
    numbers=left,                   % where to put the line-numbers
    numberstyle=\tiny\color{codegray},      % the size of the fonts that are used for the line-numbers
    stepnumber=1,                   % the step between two line-numbers. If it is 1 each line will be numbered
    numbersep=5pt,                  % how far the line-numbers are from the code
    backgroundcolor=\color{white},  % choose the background color. You must add \usepackage{color}
    showspaces=false,               % show spaces adding particular underscores
    showstringspaces=false,         % underline spaces within strings
    showtabs=false,                 % show tabs within strings adding particular underscores
    frame=single,           % adds a frame around the code
    %tabsize=2,          % sets default tabsize to 2 spaces
    captionpos=b,           % sets the caption-position to bottom
    breaklines=true,        % sets automatic line breaking
    breakatwhitespace=false,    % sets if automatic breaks should only happen at whitespace
    escapeinside={\%*}{*)},          % if you want to add a comment within your code
    commentstyle=\color{gray},
    keywordstyle=\color{blue},
}

\lstdefinestyle{SMTLIB}{
    language=Java,
    basicstyle=\footnotesize,       % the size of the fonts that are used for the code
    numbers=left,                   % where to put the line-numbers
    numberstyle=\tiny\color{codegray},      % the size of the fonts that are used for the line-numbers
    stepnumber=1,                   % the step between two line-numbers. If it is 1 each line will be numbered
    numbersep=5pt,                  % how far the line-numbers are from the code
    backgroundcolor=\color{white},  % choose the background color. You must add \usepackage{color}
    showspaces=false,               % show spaces adding particular underscores
    showstringspaces=false,         % underline spaces within strings
    showtabs=false,                 % show tabs within strings adding particular underscores
    frame=single,           % adds a frame around the code
    %tabsize=2,          % sets default tabsize to 2 spaces
    captionpos=b,           % sets the caption-position to bottom
    breaklines=true,        % sets automatic line breaking
    breakatwhitespace=false,    % sets if automatic breaks should only happen at whitespace
    escapeinside={(*}{*)},          % if you want to add a comment within your code
    commentstyle=\color{gray},
    keywordstyle=\color{blue},
    morekeywords={bvand, bvnot, concat, extract, bvxor}
}

\lstdefinestyle{KRULE}{
    %language=Java,
    %basicstyle=\footnotesize,      
    basicstyle=\scriptsize,
    backgroundcolor=\color{white},  % choose the background color. You must add \usepackage{color}
    showspaces=false,               % show spaces adding particular underscores
    showstringspaces=false,         % underline spaces within strings
    showtabs=false,                 % show tabs within strings adding particular underscores
    frame=single,           % adds a frame around the code
    %tabsize=2,          % sets default tabsize to 2 spaces
    captionpos=b,           % sets the caption-position to bottom
    breaklines=true,        % sets automatic line breaking
    breakatwhitespace=false,    % sets if automatic breaks should only happen at whitespace
    escapeinside={(*}{*)},          % if you want to add a comment within your code
    commentstyle=\color{gray},
    morecomment=[l]{//},
    keywordstyle=\color{blue},
    %morekeywords={regstate, stackmem, andBool, requires, ensures, codemem, memstate, and}
    morekeywords={andBool, requires, ensures, and, rule}
}

\lstdefinestyle{KRULEWOBORDER}{
    %language=Java,
    %basicstyle=\footnotesize,      
    basicstyle=\scriptsize,
    backgroundcolor=\color{white},  % choose the background color. You must add \usepackage{color}
    showspaces=false,               % show spaces adding particular underscores
    showstringspaces=false,         % underline spaces within strings
    showtabs=false,                 % show tabs within strings adding particular underscores
    %frame=single,           % adds a frame around the code
    %tabsize=2,          % sets default tabsize to 2 spaces
    captionpos=b,           % sets the caption-position to bottom
    breaklines=true,        % sets automatic line breaking
    breakatwhitespace=false,    % sets if automatic breaks should only happen at whitespace
    escapeinside={(*}{*)},          % if you want to add a comment within your code
    commentstyle=\color{gray},
    morecomment=[l]{//},
    keywordstyle=\color{blue},
    morekeywords={regstate, stackmem, andBool, requires, ensures, codemem, memstate, and}
}

\lstdefinestyle{SIMPRULES}{
    language=Java,
    basicstyle=\footnotesize,       % the size of the fonts that are used for the code
    backgroundcolor=\color{white},  % choose the background color. You must add \usepackage{color}
    escapeinside={(*}{*)},          % if you want to add a comment within your code
    commentstyle=\color{gray},
    morecomment=[l]{//},
}
